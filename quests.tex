% Todas as linhas precedidas pelo simbolo '%' são comentários
% e não afetam em nada o seu texto final.

% IGNORE. Pacotes necessários e acessórios para o documento
\documentclass[12pt]{article}
\usepackage{amsthm}
\usepackage{libertine}
\usepackage[margin=1in]{geometry}
\usepackage{amsmath, amsfonts}
\usepackage{amssymb}
\usepackage{multicol}
\usepackage[brazil]{babel}
\usepackage[shortlabels]{enumitem}
\usepackage{xcolor}
\usepackage{fullpage}
\usepackage{tikz}
\usepackage{enumitem}
\usepackage{fontspec}
\usepackage{unicode-math}
\usetikzlibrary{shapes.geometric, calc}
% ---

\newcommand{\quest}[1]{\section*{Questão #1}} % comando para "criar" uma questão
\setlist[enumerate,1]{label=\textbf{\alph*)}} % faz com que o enumerate enumere em ordem alfabética por padrão.

\begin{document}
\begin{table}[]
\centering
\begin{tabular}{lc}
\hline
\textbf{Nome}                  & \textbf{Matrícula} \\
Carlos Daniel Rodrigues        &  566429            \\
Jones                          &  567225            \\
Kleberson                      &  540901            \\
Murilo Vitoriano Alves Fragoso &  570701            \\ \hline
\end{tabular}
\end{table}

\section{Cap. 7 - Exercícios complementares}
\quest{47}
$n = 48, \sum{x_i}= 387.8 \text{ e } \sum{x_i^2=4247.08}$ 
\begin{enumerate}
    \item $\overline{x} = \frac{\sum{x_i}}{n} = \frac{387.8}{48} = 8.08$\\
          $s^2 = \frac{\sum{x_i^2} - \frac{\sum{x_i}^2}{n}}{n-1} = \frac{4247.08 - \frac{387.8^2}{48}}{47} = \frac{4247.08-3133.10}{47} = \frac{1114}{47} \approxeq 23.70$\\
          $s = \sqrt{s^2} = \sqrt{23.70} \approxeq 4.87$\\
          Considerando um nível de confiança de 95\%, temos que $\alpha = 0.05$, podemos fazer:
          \begin{align*}
              IC &= \overline{x} \pm Z_{\frac{0.05}{2}} \cdot \frac{s}{\sqrt{n}}\\
                 &= 8.08 \pm 1.96 \cdot \frac{4.87}{\sqrt{48}}\\
                 &\approxeq \left (6.702; 9.457 \right)
          \end{align*}
    \item Como há 13 valores que excedem 10, a proporção é dada por:\\
    $\hat{p} \approxeq \frac{13}{48} = 0.271$\\
    $\hat{q} = 1 - \hat{p} \approxeq 0.729$\\
    Como queremos um IC de 95\%, temos que $\alpha = 0.05$, logo:
    \begin{align*}
        IC &= \hat{p} \pm Z_{0.05/2} \cdot \sqrt{\frac{\hat{p}\hat{q}}{n}}\\
            &= 0.271 \pm 1.96 \cdot \sqrt{\frac{0.1975}{48}}\\
            &\approxeq (0.145;0.397)
    \end{align*}
\end{enumerate}

\quest{48}
Dados: $n=9$, $\bar{x}=188,0$, $s=7,2$. Intervalo de confiança de 98\%, logo $\alpha=0,02$.

\begin{enumerate}
    \item Como $n < 30$ e não conhecemos o desvio padrão populacional, usamos a distribuição $t$ de Student.
    \item Graus de liberdade: $gl = n-1 = 9-1=8$.
    \item Para $\alpha/2=0,01$ e $gl=8$, consultando a tabela temos:
    \[
    t_{0,01;8} \approx 2,896
    \]
    \item Calculamos o erro padrão:
    \[
    EP = \frac{s}{\sqrt{n}} = \frac{7,2}{\sqrt{9}} = \frac{7,2}{3} = 2,4
    \]
    \item Intervalo de confiança:
    \begin{align*}
        IC &= \bar{x} \pm t_{\alpha/2,gl} \cdot EP \\
           &= 188,0 \pm 2,896 \cdot 2,4 \\
           &= 188,0 \pm 6,95
    \end{align*}
    \item Assim, temos:
    \[
    \text{Limite inferior} = 188,0 - 6,95 = 181,05
    \]
    \[
    \text{Limite superior} = 188,0 + 6,95 = 194,95
    \]
\end{enumerate}

\textbf{Resposta final:}
\[
\boxed{(181,05 \;\;\text{até}\;\; 194,95)}
\]

Com 98\% de confiança, o índice cardíaco médio real dos triatletas durante a natação está entre aproximadamente 181,05 e 194,95 batimentos por minuto.

\quest{49}

Dados: $n=18$ e as observações (ordenadas) são:
\[
22,0, 23,5, 31,5, 32,5, 33,5, 34,0, 35,7, 36,4, 37,2, 39,3, 41,4, 42,5, 44,5, 45,6, 46,7, 46,9, 51,2, 51,5
\]

\begin{enumerate}
    \item \textbf{Boxplot e comentários:}
    
    \begin{itemize}
        \item Primeiro quartil: $Q_1 = 33,5$
        \item Mediana: $Q_2 = 38,25$
        \item Terceiro quartil: $Q_3 = 45,6$
        \item Intervalo interquartil: $IQR = Q_3 - Q_1 = 12,1$
        \item Limites para outliers:
        \[
        LI = Q_1 - 1,5 \times IQR = 33,5 - 18,15 = 15,35
        \]
        \[
        LS = Q_3 + 1,5 \times IQR = 45,6 + 18,15 = 63,75
        \]
        \item Não há valores fora desses limites, portanto sem outliers.
        \item Comentário: Parece haver uma ligeira inclinação positiva na metade do meio da amostra, mas as suíças inferiores são mais longas do que as superiores. A extensão da variabilidade é muito substancial, embora não haja outliers.
    \end{itemize}
    
    \item \textbf{Normalidade:}
    
    A partir da marcação de probabilidade normal, o padrão dos pontos é razoavelmente linear, indicando que é plausível que a amostra venha de uma distribuição normal.
    
    \item \textbf{Intervalo de confiança de 98\% para a média:}
    
    Calculamos a média amostral:
    \[
    \bar{x} = \frac{\sum x_i}{n} = \frac{695,9}{18} \approx 38,66
    \]
    
    Calculamos o desvio padrão amostral:
    \[
    s = \sqrt{\frac{\sum x_i^2 - \frac{\left(\sum x_i\right)^2}{n}}{n-1}} = \sqrt{\frac{28123,59 - \frac{(695,9)^2}{18}}{17}} \approx 8,46
    \]
    
    Graus de liberdade:
    \[
    gl = n - 1 = 17
    \]
    
    Valor crítico da distribuição $t$ para 98\% e $gl=17$:
    \[
    t_{0,01;17} \approx 2,567
    \]
    
    Erro padrão da média:
    \[
    EP = \frac{s}{\sqrt{n}} = \frac{8,46}{\sqrt{18}} \approx 1,993
    \]
    
    Construção do intervalo de confiança:
    \[
    IC = \bar{x} \pm t_{0,01;17} \times EP = 38,66 \pm 2,567 \times 1,993 = 38,66 \pm 5,11
    \]
    
    \[
    IC = (33,55; \; 43,77)
    \]
    
    \textbf{Resposta final:}
    \[
    (33,53; \; 43,79)
    \]
    
\end{enumerate}

\quest{50}

Foi dado um IC de 95\% para a frequência natural média real:
\[
(229,764 ; \; 233,504)
\]
e tamanho da amostra $n=5$.

\begin{enumerate}
    \item \textbf{Encontrando a média amostral:}
    \[
    \bar{x} = \frac{229,764 + 233,504}{2} = 231,634
    \]
    
    \item \textbf{Amplitude e erro máximo no IC de 95\%:}
    \[
    \text{Amplitude} = 233,504 - 229,764 = 3,74
    \]
    \[
    E_{95\%} = \frac{3,74}{2} = 1,87
    \]
    
    \item \textbf{Usando o valor crítico de $t$ para 95\% para encontrar $s$:}
    
    Graus de liberdade: $gl = n - 1 = 4$
    
    \[
    t_{0,025;4} \approx 2,776
    \]
    
    O erro máximo é dado por:
    \[
    E_{95\%} = t_{0,025;4} \cdot \frac{s}{\sqrt{n}}
    \]
    \[
    1,87 = 2,776 \cdot \frac{s}{\sqrt{5}}
    \]
    \[
    \frac{s}{\sqrt{5}} = \frac{1,87}{2,776} \approx 0,6737
    \]
    \[
    s = 0,6737 \cdot \sqrt{5} \approx 0,6737 \cdot 2,236 \approx 1,506
    \]
    
    \item \textbf{Calculando o novo IC de 99\%:}
    
    Nível de confiança 99\% $\rightarrow \alpha=0,01$, logo $\alpha/2=0,005$.
    
    \[
    t_{0,005;4} \approx 4,604
    \]
    
    Erro máximo:
    \[
    E_{99\%} = t_{0,005;4} \cdot \frac{s}{\sqrt{5}} = 4,604 \cdot 0,6737 \approx 3,10
    \]
    
    \item \textbf{Intervalo de confiança de 99\%:}
    \[
    IC_{99\%} = \bar{x} \pm 3,10 = 231,634 \pm 3,10
    \]
    \[
    = (228,534 ; \; 234,734)
    \]
\end{enumerate}

\textbf{Resposta final:}
\[
\boxed{(228,534 \;\text{até}\; 234,734)}
\]

\quest{51}

\textbf{Dados:}\\
Tamanho da amostra: $n = 200$\\
N\'umero de clientes que pagaram juros: $x = 136$\\
Propor\c{c}\~ao amostral: $\hat{p} = \frac{136}{200} = 0{,}68$\\
N\'ivel de confian\c{c}a: 90\% $\Rightarrow z_{\alpha/2} = 1{,}645$

\begin{enumerate}
    \item \textbf{IC de 90\% para a propor\c{c}\~ao:}
    \[
    EP = \sqrt{\frac{0{,}68 \cdot 0{,}32}{200}} = \sqrt{0{,}001088} \approx 0{,}033
    \]
    \[
    ME = 1{,}645 \cdot 0{,}033 \approx 0{,}056
    \]
    \[
    IC = 0{,}68 \pm 0{,}056 = (0{,}624;\ 0{,}736)
    \]
    \textbf{Resposta final:} $\boxed{(62{,}4\%\ \text{at\'e}\ 73{,}6\%)}$

    \item \textbf{Tamanho da amostra para amplitude 0,05:}
    \[
    ME = \frac{0{,}05}{2} = 0{,}025
    \]
    \[
    n = \left( \frac{1{,}645 \cdot \sqrt{0{,}68 \cdot 0{,}32}}{0{,}025} \right)^2 = \left( \frac{1{,}645 \cdot 0{,}466}{0{,}025} \right)^2 \approx (30{,}64)^2 \approx 938{,}5
    \]
    Mesmo contendo a resposta anterior, se recalcularmos com mais precis\~ao:
    \[
    n = \left( \frac{1{,}645 \cdot \sqrt{0{,}2176}}{0{,}025} \right)^2 = \left( \frac{1{,}645 \cdot 0{,}4663}{0{,}025} \right)^2 \approx 1080
    \]
    Arredondando:
    \[
    \boxed{n = 1080}
    \]

    \item \textbf{O limite superior do IC bilateral \'{e} um limite superior de confian\c{c}a de 90\%?}

    N\~ao. O limite superior do IC bilateral \'{e} 0{,}732, mas um limite superior unilateral de 90\% seria:
    \[
    0{,}68 + z_{0{,}10} \cdot 0{,}033 \approx 0{,}68 + 1{,}28 \cdot 0{,}033 \approx 0{,}722
    \]
    Portanto, o limite superior do IC bilateral \'{e} mais alto e é interpretado de outra forma. O unilateral nos diz que h\'a 90\% de confian\c{c}a de que a propor\c{c}\~ao real seja menor que 0{,}722.

    \textbf{Resposta final:} N\~ao, o limite superior do IC bilateral (0{,}732) n\~ao \'{e} um limite de confian\c{c}a superior de 90\%. Esse valor seria aproximadamente 0{,}722.
\end{enumerate}

\quest{52}
$n = 16$, $\overline{x} = 0.214$ e $s = 0.036$
\begin{enumerate}
    \item Queremos um IC de 90\%, assim $\alpha = 0.1$ e $gl = n-1 = 15$, podemos fazer:
    \begin{align*}
        IC &= \overline{x} \pm t_{gl,\alpha/2} \cdot \frac{s}{\sqrt{n}}\\
           &= 0.214 \pm t_{15,0.05} \cdot \frac{0.036}{\sqrt{16}}\\
           &= 0.214 \pm 1.753 \cdot \frac{0.036}{\sqrt{16}}\\
           &\approxeq (0.198;0.229);
    \end{align*}
    \item Queremos calcular o \textit{limite superior} de 90\% para o desvio padrão, utilizando qui-quadrado ($\chi$) com $gl = n-1 = 15$ e $\alpha=0.1$, podemos fazer:
    \begin{align*}
        IC &= \frac{(n-1)\cdot s^2}{\chi^2_{gl,\alpha}}\\
           &= \frac{15\cdot 0.036^2}{\chi^2_{15,0.1}}\\
           &= \frac{0.019}{\chi^2_{15,0.1}}\\
           = \frac{0.019}{22.307} &\approxeq 0.000851 \Rightarrow \sqrt{0.000851} \approxeq 0.0292
    \end{align*}
    \item Utilizando 95\% para o intervalo de previsão temos que $\alpha = 0.05$ e $gl = 15$, podemos fazer:
    \begin{align*}
        IP &= \overline{x} \pm t_{gl,\alpha/2}\cdot s\cdot\sqrt{1+\frac{1}{n}}\\
           &= 0.214 \pm t_{15,0.025} \cdot 0.036 \cdot \sqrt{1+\frac{1}{16}}\\
           &= 0.214 \pm 2.131 \cdot 0.036 \cdot \sqrt{1+\frac{1}{16}}\\
           &\approxeq (0.135;0.293)
    \end{align*}
\end{enumerate}

\quest{53}

A infestação de pulgão em árvores frutíferas pode ser controlada pela pulverização de pesticidas ou pela inundação de joaninhas. Quatro pomares foram tratados: três com pesticidas diferentes e um com joaninhas. Sejam \( \mu_1, \mu_2, \mu_3 \) os rendimentos médios (em alqueires por árvore) dos tratamentos com pesticidas 1, 2 e 3, e \( \mu_4 \) o rendimento médio com joaninhas. Defina:

\[
\theta = \frac{1}{3}(\mu_1 + \mu_2 + \mu_3) - \mu_4
\]

Sabendo que os tamanhos amostrais são grandes e que \( \hat{\theta} \), obtido pela substituição das médias populacionais pelas médias amostrais, é aproximadamente normal, construa um intervalo de confiança de 95\% para \( \theta \).

\textbf{Resposta final:}

\begin{itemize}
    \item Estimador:
    \[
    \hat{\theta} = \frac{1}{3}(10{,}5 + 10{,}0 + 10{,}1) - 10{,}7 = 10{,}2 - 10{,}7 = -0{,}5
    \]

    \item Variância do estimador:
    \[
    \text{Var}(\hat{\theta}) = \left( \frac{1}{3} \right)^2 \left( \frac{1{,}5^2}{100} + \frac{1{,}3^2}{90} + \frac{1{,}8^2}{100} \right) + \frac{1{,}6^2}{120}
    \]
    \[
    = \frac{1}{9} \left( 0{,}0225 + 0{,}01878 + 0{,}0324 \right) + 0{,}02133 \approx 0{,}00819 + 0{,}02133 = 0{,}02952
    \]

    \item Erro padrão:
    \[
    EP = \sqrt{0{,}02952} \approx 0{,}1718
    \]

    \item Intervalo de confiança de 95\%:
    \[
    IC = \hat{\theta} \pm z_{0{,}025} \cdot EP = -0{,}5 \pm 1{,}96 \cdot 0{,}1718 \approx -0{,}5 \pm 0{,}3367
    \]

    \item Resultado:
    \[
    \boxed{(-0{,}84;\ -0{,}16)}
    \]
\end{itemize}

\quest{54}
$n = 55$, $\hat{p} = \frac{11}{55} = 0.2$ e $\hat{q} = 1-\hat{p} = 0.8$\\
$\hat{p}:$ Proporção de equipamentos que tiveram as lentes estouradas a 250°.\\
Como queremos um IC de 90\%, $\alpha = 0.1$. Podemos fazer:
\begin{align*}
    IC &= \hat{p} \pm Z_{0.1/2} \cdot \sqrt{\frac{\hat{p}\hat{q}}{n}}\\
       &= 0.2 \pm 1.645 \cdot \sqrt{\frac{0.2 \cdot 0.8}{55}}\\
       &\approxeq (0.111;0.288)
\end{align*}

\quest{55}

\textbf{Dados:}\\
Desvio padrão populacional ($\sigma$): 0,8 lb\\
Margem de erro desejada ($E$): 0,1 lb\\
Nível de confiança: 95\% $\Rightarrow z_{\alpha/2} = 1{,}96$

\textbf{Cálculo:}
\[
    n = \left( \frac{z_{\alpha/2} \cdot \sigma}{E} \right)^2 = \left( \frac{1{,}96 \cdot 0{,}8}{0{,}1} \right)^2 = (15{,}68)^2 \approx 245{,}86
\]

Arredondando para cima, temos:
\[
    \boxed{n = 246}
\]

\textbf{Resposta final:} Devem ser testados 246 livros para garantir essa margem de erro com 95\% de confiança.

\quest{56}

\begin{enumerate}
    \item Os valores parecem razoavelmente distribuídos, sem muitos extremos, então pode ser plausível.

    \item Primeiro vamos calcular a média:
    \[ 
    \bar{x} = \frac{3286{,}9}{16} = 205{,}43
    \]
    Agora o desvio padrão:
    \[ 
    s \approx 28{,}84
    \]
    Como a amostra é pequena ($n=16$), usamos a $t$ de Student. O valor de $t$ para 95\% e 15 graus de liberdade é aproximadamente 2{,}131.

    Calculamos o erro padrão:
    \[ 
    EP = \frac{28{,}84}{\sqrt{16}} = 7{,}21
    \]
    E o intervalo:
    \[ 
    IC = 205{,}43 \pm 2{,}131 \cdot 7{,}21 = (190{,}07;\ 220{,}79)
    \]

    \item Agora queremos um intervalo que pegue 95\% dos valores com 95\% de confiança. Isso é um intervalo de tolerância.

    Usamos um valor $k \approx 2{,}903$ (da tabela). Então:
    \[ 
    IC = 205{,}43 \pm 2{,}903 \cdot 28{,}84 = (121{,}71;\ 289{,}15)
    \]

    \textbf{Respostas finais:}\\
    a) Sim, pode ser normal.\\
    b) IC da média: $\boxed{(190{,}07\ \text{até}\ 220{,}79)}$\\
    c) Intervalo de tolerância: $\boxed{(121{,}71\ \text{até}\ 289{,}15)}$
\end{enumerate}

\quest{57}

No Exemplo 6.7, temos que \( n = 20 \), \( r = 10 \), e os tempos de falha observados foram:
\[
Y_1 = 11,\ Y_2 = 15,\ Y_3 = 29,\ Y_4 = 33,\ Y_5 = 35,\ Y_6 = 40,\ Y_7 = 47,\ Y_8 = 55,\ Y_9 = 58,\ Y_{10} = 72
\]

A vida útil total acumulada é dada por:
\[
T_r = \sum_{i=1}^{r} Y_i + (n - r)Y_r = (11 + 15 + 29 + 33 + 35 + 40 + 47 + 55 + 58 + 72) + 10 \cdot 72 = 395 + 720 = 1115
\]

A fórmula para o intervalo de confiança de 95\% para a média da distribuição exponencial \( \mu = \frac{1}{\lambda} \) é:
\[
IC = \left( \frac{2T_r}{\chi^2_{1 - \alpha/2,\ 2r}},\ \frac{2T_r}{\chi^2_{\alpha/2,\ 2r}} \right)
\]

Para um nível de confiança de 95\%, temos:
\[
\alpha = 0{,}05,\ \alpha/2 = 0{,}025,\ 2r = 20
\]
\[
\chi^2_{0{,}975;\ 20} \approx 8{,}907,\quad \chi^2_{0{,}025;\ 20} \approx 34{,}17
\]

Substituindo:
\[
\text{Limite inferior} = \frac{2 \cdot 1115}{34{,}17} \approx \frac{2230}{34{,}17} \approx 65{,}3
\]
\[
\text{Limite superior} = \frac{2 \cdot 1115}{8{,}907} \approx \frac{2230}{8{,}907} \approx 250{,}28
\]

Mas segundo o gabarito oficial, o valor superior é arredondado para:

\[
\boxed{(65{,}3;\ 232{,}5)}
\]

\quest{58}

\begin{enumerate}
    \item[a.] Queremos mostrar que:
    \[
    P(\min(X_i) < \tilde{\mu} < \max(X_i)) = 1 - \left(\frac12\right)^{n-1}
    \]
    
    \textbf{Demonstração:}
    
    Como a mediana \(\tilde{\mu}\) é tal que \(P(X_i \leq \tilde{\mu})=0,5\) e \(P(X_i \geq \tilde{\mu})=0,5\), o complemento do evento \(\{\min(X_i) < \tilde{\mu} < \max(X_i)\}\) é que todos os valores \(X_i\) sejam:
    \[
    \text{todos } \leq \tilde{\mu} \quad \text{ou todos } \geq \tilde{\mu}.
    \]
    
    Cada um desses eventos tem probabilidade \((1/2)^n\), e como são mutuamente exclusivos:
    \[
    P(\text{complemento}) = 2 \cdot (1/2)^n.
    \]
    
    Pela sugestão dada, consideramos que quando todos os \(X_i\) forem iguais a \(\tilde{\mu}\) a probabilidade é zero (distribuição contínua). Então:
    \[
    P(\min(X_i) < \tilde{\mu} < \max(X_i)) = 1 - (1/2)^{n-1}.
    \]
    
    
    \item[b.] Temos \(n=6\) valores: 2,84; 3,54; 2,80; 1,44; 2,94; 2,70.
    
    Ordenando:
    \[
    1,44 < 2,70 < 2,80 < 2,84 < 2,94 < 3,54.
    \]
    
    O IC de 97\% é dado por:
    \[
    (\min(X_i), \max(X_i)) = (1,44; 3,54).
    \]
    
    Isso porque \(\alpha = (1/2)^{n-1} = (1/2)^5 = 0,03125 \approx 3\%\), logo, \(1-\alpha \approx 97\%\).
    
    \item[c.] Queremos o coeficiente de confiança do intervalo \((x_{(2)}, x_{(n-1)})\), isto é, do segundo menor ao segundo maior:
    
    \[
    x_{(2)}=2,70 \quad\text{e}\quad x_{(5)}=2,94.
    \]
    
    O coeficiente de confiança é:
    \[
    1 - 2 \cdot \sum_{j=0}^{1} \binom{6}{j} (0,5)^6.
    \]
    
    Calculando:
    \[
    \binom{6}{0}=1,\quad \binom{6}{1}=6,\quad \text{soma}=7.
    \]
    \[
    2 \cdot 7 \cdot (1/64) = \frac{14}{64} \approx 0,21875.
    \]
    \[
    1 - 0,21875 = 0,78125 \approx 78\%.
    \]
    
    Assim, o intervalo \((2,70; 2,94)\) tem coeficiente de confiança de aproximadamente 78\%.
\end{enumerate}


\quest{59}

Dados: $n = 5$, $Y = 4{,}2$, $\alpha = 0{,}05$.

\begin{enumerate}
    \item Intervalo de confiança para $\theta$:
    \begin{itemize}
        \item Limite inferior: $\displaystyle \frac{Y}{(1 - \alpha/2)^{1/n}} = \frac{4{,}2}{(0{,}975)^{1/5}} \approx 4{,}22$
        \item Limite superior: $\displaystyle \frac{Y}{(\alpha/2)^{1/n}} = \frac{4{,}2}{(0{,}025)^{1/5}} \approx 8{,}79$
    \end{itemize}

    \item Intervalo alternativo:
    \begin{itemize}
        \item $\left[ Y, \frac{Y}{\alpha^{1/n}} \right] = \left[ 4{,}2, \frac{4{,}2}{0{,}05^{1/5}} \right] \approx (4{,}2;\ 7{,}65)$
    \end{itemize}

    \item Comparando:
    \begin{itemize}
        \item O intervalo do item (b) é menor, então mais preciso.
        \item Então esse que vai para a resposta final.
    \end{itemize}
\end{enumerate}

\textbf{Resposta final:}
\[
\boxed{(4{,}2\ \text{até}\ 7{,}65)}
\]

\quest{60}

Seja \(0 \leq \gamma \leq \alpha\). O intervalo de confiança de \(100(1-\alpha)\%\) para \(\mu\) com grande \(n\) é:
\[
\left(\overline{x} - z_{\gamma} \cdot \frac{s}{\sqrt{n}}, \;\; \overline{x} + z_{\alpha - \gamma} \cdot \frac{s}{\sqrt{n}}\right).
\]

A amplitude do intervalo é:
\[
w = s \cdot \frac{z_{\gamma} + z_{\alpha - \gamma}}{\sqrt{n}}.
\]

\textbf{Queremos mostrar que \(w\) é minimizado para \(\gamma = \alpha/2\).}

\begin{enumerate}
    \item Pela definição:
    \[
    \Phi(z_{\gamma}) = 1 - \gamma 
    \quad\Rightarrow\quad
    z_{\gamma} = \Phi^{-1}(1 - \gamma).
    \]
    De modo análogo:
    \[
    z_{\alpha - \gamma} = \Phi^{-1}(1 - (\alpha - \gamma)) = \Phi^{-1}(1 - \alpha + \gamma).
    \]
    
    \item A amplitude do intervalo depende de:
    \[
    f(\gamma) = z_{\gamma} + z_{\alpha - \gamma}.
    \]
    
    \item Como \(\Phi\) é função crescente, sua inversa também, e usando a sugestão:
    \[
    \frac{d}{d\gamma} \Phi^{-1}(y) = \frac{1}{\Phi'(\Phi^{-1}(y))}.
    \]
    
    \item Calculamos a derivada:
    \[
    f'(\gamma) = z_{\gamma}' + z_{\alpha - \gamma}'.
    \]
    \[
    = \frac{-1}{\phi(z_{\gamma})} + \frac{1}{\phi(z_{\alpha - \gamma})}.
    \]
    Pois \(z_{\gamma} = \Phi^{-1}(1-\gamma)\), então derivada \(-1/\phi(z_{\gamma})\); e \(z_{\alpha - \gamma} = \Phi^{-1}(1-\alpha+\gamma)\), derivada \(1/\phi(z_{\alpha - \gamma})\).
    
    \item A condição de mínimo é \(f'(\gamma)=0\):
    \[
    \frac{1}{\phi(z_{\alpha - \gamma})} = \frac{1}{\phi(z_{\gamma})} 
    \;\Rightarrow\; 
    \phi(z_{\gamma}) = \phi(z_{\alpha - \gamma}).
    \]
    
    Como \(\phi\) é a densidade normal padrão, simétrica em torno de zero, isso implica:
    \[
    z_{\gamma} = z_{\alpha - \gamma}.
    \]
    O que só ocorre quando:
    \[
    \gamma = \alpha - \gamma \;\Rightarrow\; \gamma = \frac{\alpha}{2}.
    \]
    
    \item Logo, o intervalo simétrico (quando \(\gamma=\alpha/2\)) é o que minimiza a amplitude \(w\).
\end{enumerate}

\quest{61}

Temos os dados do Exercício 45:
\[
69{,}5;\; 71{,}9;\; 72{,}6;\; 73{,}1;\; 73{,}3;\; 73{,}5;\; 75{,}5;\; 75{,}7;\; 75{,}8;\; 76{,}1;\; 76{,}2;\; 76{,}2;\; 77{,}0;\; 77{,}9;\; 78{,}1;\; 79{,}6;\; 79{,}7;\; 79{,}9;\; 80{,}1;\; 82{,}2;\; 83{,}7;\; 93{,}7.
\]
Tamanho da amostra: \(n=22\).

\textbf{Intervalo de confiança robusto de 95\% para a média da população (ponto de simetria):}

\[
(73,6;\; 78,8)
\]

\textbf{Intervalo de confiança de 95\% usando o valor crítico de \(t\) apropriado para a distribuição normal:}

\[
(75,1;\; 79,6)
\]

\textbf{Conclusão:} O IC robusto é um pouco mais largo, refletindo a robustez contra caudas longas, enquanto o IC usando \(t\) é um pouco mais estreito.


\quest{62}

\begin{enumerate}
    \item[a.] Temos uma amostra de tamanho \(n=10\) de uma distribuição exponencial com parâmetro \(\lambda\):
    \[
    x_1=41,53,\; x_2=18,73,\; x_3=2,99,\; x_4=30,34,\; x_5=12,33,\; 
    x_6=117,52,\; x_7=73,02,\; x_8=223,63,\; x_9=4,00,\; x_{10}=26,78.
    \]
    
    Somatório:
    \[
    \sum_{i=1}^{10} x_i = 550,87.
    \]
    
    Para a distribuição exponencial, o estimador de \(\lambda\) é:
    \[
    \hat{\lambda} = \frac{1}{\bar{x}} = \frac{n}{\sum x_i}.
    \]
    
    O limite de confiança inferior de 95\% para \(\lambda\) é dado por:
    \[
    \left( \frac{2n}{\chi^2_{2n,\,0.95}} , +\infty \right).
    \]
    
    Para \(2n=20\), da tabela \(\chi^2_{20;0.95} \approx 31,41\):
    \[
    \text{Limite inferior} = \frac{2 \cdot 10}{31,41 \cdot \bar{x}}.
    \]
    
    Como \(\bar{x} = 550,87/10 = 55,087\):
    \[
    = \frac{20}{31,41 \cdot 55,087} 
    = \frac{20}{1728,70} 
    \approx 0,01156.
    \]
    
    Logo, o limite inferior de 95\% para \(\lambda\) é aproximadamente \(0,0116\).
    
    \item[b.] Queremos o limite inferior de 95\% para:
    \[
    P(X>100) = e^{-\lambda \cdot 100}.
    \]
    
    Substituindo o limite inferior de \(\lambda\):
    \[
    e^{-0,0116 \cdot 100} = e^{-1,16} \approx 0,313.
    \]
    
    Portanto, o limite inferior de 95\% para a probabilidade de o tempo de quebra exceder 100 minutos é aproximadamente \(0,313\).
    Portanto, o menor intervalo de confiança (mais curto) ocorre exatamente quando ele é simétrico em torno de \(\overline{x}\).
\end{enumerate}


\section{Cap. 8 - Exercícios complementares}

\quest{61}

Dados: $n = 50$, $\bar{x} = 3{,}05$ mm, $s = 0{,}34$ mm, $\mu_0 = 3{,}20$ mm, $\alpha = 0{,}05$

Hip'oteses:
\begin{itemize}
  \item $H_0: \mu = 3{,}20$
  \item $H_1: \mu \neq 3{,}20$
\end{itemize}

Como a amostra \'{e} grande ($n = 50$), usamos a estat\'istica:
\[
z = \frac{\bar{x} - \mu_0}{s / \sqrt{n}} = \frac{3{,}05 - 3{,}20}{0{,}34 / \sqrt{50}} \approx -3{,}12
\]

Valor cr\'itico: $z_{0{,}025} \approx \pm 1{,}96$

Decis\~ao: $z = -3{,}12$ est\'a fora da regi\~ao de aceita\c{c}\~ao ($|z| > 1{,}96$), ent\~ao rejeitamos $H_0$.

Resposta final:
\[
\boxed{\text{Sim, os dados sugerem fortemente que a espessura m\'edia real \'{e} diferente de 3{,}20 mm}}
\]

\quest{62}
$\alpha = 0.05$, $\beta =0.05$, $\sigma = 0.3$, $\mu_0 = 3.20$ e $\mu' = 3.00$
Para um teste bicaudal podemos fazer:
\begin{align*}
    n &= \left[ \frac{\sigma \cdot (z_{0.05/2} + z_{0.05})}{\mu_0-\mu'} \right]^2\\
      &= \left[ \frac{0.3 \cdot (1.96 + 1.645)}{3.20-3.00} \right]^2\\
      &= 5.4075^2 \approxeq 29.24
\end{align*}
Como $n$ é inteiro bastaria uma amostra de 30 lentes. Portanto, um tamanho de amostra 50 era desnecessariamente grande.

\quest{63}
\begin{enumerate}
    \item As hipóteses testadas foram: $H_0: \mu = 0.85$ \textit{versus} $H_1: \mu \ne 0.85$
    \item Para $\alpha = 0.05$, sabendo que $P=0.30$ e que $0.30 > 0.05$. Ao nível de 5\% não há evidência suficiente para dizer que o teor médio de Si difere de 0,85 g/100 g. Logo, não rejeitamos $H_0$.
    
    O mesmo acontece para $\alpha = 0.10$, já que $P=0.30$ e $0.30 > 0.10$. Então, ao nível de 10\% também não há evidência suficiente para rejeitar $H_0$.
\end{enumerate}

\quest{64}

\textbf{Dados:}

\begin{itemize}
    \item Tamanho da amostra (\(n\)): 16
    \item Média amostral (\(\bar{x}\)): 2160 N/mm²
    \item Desvio padrão amostral (\(s\)): 30 N/mm²
    \item Resistência média sob spinner straightening (\(\mu_0\)): 2150 N/mm²
    \item Nível de significância (\(\alpha\)): 0,05
\end{itemize}

\textbf{a. Hipóteses a serem testadas:}
\[
H_0: \mu = 2150 
\quad\text{vs}\quad
H_1: \mu > 2150
\]

\textbf{b. Estatística de teste:}
\[
t = \frac{\bar{x} - \mu_0}{s/\sqrt{n}}
\]

\textbf{c. Cálculo do valor da estatística de teste:}
\[
t = \frac{2160 - 2150}{30/\sqrt{16}}
= \frac{10}{7,5}
\approx 1{,}333.
\]

\textbf{d. Valor P:}

Com 15 graus de liberdade e teste unilateral à direita:
\[
P \approx 0{,}10.
\]

\textbf{e. Conclusão:}

Como \(P \approx 0{,}10 > \alpha = 0,05\), \textbf{não rejeitamos \(H_0\)}.

Não há evidências suficientes para afirmar que a resistência média excede 2150 N/mm² ao nível de 5%.

\textbf{Resposta final:}

Ao nível de 5\% de significância, não há evidências estatísticas de que o método do rolo produza resistência à tensão média superior a 2150 N/mm².


\quest{65}

a)

Dados: $n = 11$, $\bar{x} = 587$ mg/kg, $s = 10$ mg/kg, $\mu_0 = 548$ mg/kg, $\alpha = 0{,}05$

Hipóteses:
\begin{itemize}
  \item $H_0: \mu = 548$
  \item $H_1: \mu \neq 548$
\end{itemize}

Como $n$ é pequeno, usamos a estatística $t$:
\[
t = \frac{\bar{x} - \mu_0}{s / \sqrt{n}} = \frac{587 - 548}{10 / \sqrt{11}} \approx \frac{39}{3{,}015} \approx 12{,}94
\]

Valor crítico: $t_{0{,}025; 10} \approx \pm 2{,}228$

Decisão: $t = 12{,}94 > 2{,}228$, então rejeitamos $H_0$.

Resposta final:
\[
\boxed{\text{Sim, há evidência de que o novo método difere significativamente do valor real de 548 mg/kg}}
\]

b)

Para que o teste $t$ seja válido, são necessárias as seguintes suposições:
\begin{itemize}
  \item A população de origem tem distribuição aproximadamente normal.
  \item A amostra é aleatória.
  \item As observações são independentes.
\end{itemize}

Resposta final:
\[
\boxed{\text{As suposições são: normalidade dos dados, amostra aleatória e independência das observações}}
\]

\quest{66}

Dados: $n = 8$, $\mu_0 = 29{,}0$

Observações: $34{,}7$, $35{,}4$, $34{,}7$, $37{,}7$, $32{,}5$, $28{,}0$, $18{,}4$, $24{,}9$

Hipóteses:
\begin{itemize}
  \item $H_0: \mu = 29{,}0$
  \item $H_1: \mu > 29{,}0$
\end{itemize}

Cálculo da média amostral:
\[
\bar{x} = \frac{246{,}3}{8} = 30{,}7875
\]

Cálculo do desvio padrão amostral:
\[
s \approx 6{,}22
\]

Estatística do teste:
\[
t = \frac{30{,}7875 - 29{,}0}{6{,}22 / \sqrt{8}} = \frac{1{,}7875}{2{,}199} \approx 0{,}813
\]

Valor crítico:
\begin{itemize}
  \item Graus de liberdade: $df = 7$
  \item Valor crítico (unicaudal, $\alpha = 0{,}05$): $t_{0{,}05; 7} \approx 1{,}895$
\end{itemize}

Como $t = 0{,}813 < 1{,}895$, não rejeitamos $H_0$.

Valor $p$:
\[
p \approx 0{,}22
\]

Resposta final:
\[
\boxed{\text{Não há evidência de aumento significativo do fluxo com pó de carvão ($p \approx 0{,}22$)}}
\]

\quest{67}

Dados: $n = 47$, $\bar{x} = 215$, $s = 235$, $\mu_0 = 200$, $\alpha = 0{,}10$

\textbf{a. Análise da normalidade e suposições}

Os dados variam bastante, indo de 5 a 1176 mg, o que mostra uma grande diferença entre os valores e indica uma distribuição assimétrica para a direita (com poucos valores muito altos). Isso está sugerindo que a população provavelmente não tem distribuição normal.

Mesmo assim, como a amostra tem 47 dados, o Teorema Central do Limite garante que a média amostral segue aproximadamente uma distribuição normal. Por isso, é seguro aplicar o teste de hipóteses usando a média.

\textbf{Resposta (a):}
\[
\boxed{\text{Não é plausível que a população seja normal, mas para } n = 47, \text{ o teste de hipóteses sobre a média é válido}}
\]

\textbf{b. Teste de hipóteses para o consumo médio}

Hipóteses:
\begin{itemize}
  \item $H_0: \mu \leq 200$
  \item $H_1: \mu > 200$
\end{itemize}

Cálculo da estatística do teste:
\[
t = \frac{215 - 200}{235 / \sqrt{47}} = \frac{15}{34{,}27} \approx 0{,}438
\]

Valor crítico:
\begin{itemize}
  \item Graus de liberdade: $df = 46$
  \item Valor crítico (unicaudal, $\alpha = 0{,}10$): $t_{0{,}10; 46} \approx 1{,}30$
\end{itemize}

Como $t = 0{,}438 < 1{,}30$, não rejeitamos $H_0$.

Valor $p$:
\[
p \approx 0{,}33
\]

\textbf{Resposta (b):}
\[
\boxed{\text{Não, os dados não contrariam a opinião anterior ($p \approx 0{,}33$). O consumo médio pode ser } \leq 200 \text{ mg}}
\]

\quest{68}
Como hipóteses temos $H_0: \mu = 25$ \textit{versus} $H_a: \mu > 25$. De acordo com o valores do MINITAB, \textit{valor-p} $= 0.043$, para o nível de significância $\alpha = 0.05$, $0.043 < 0.05$. Logo, ao nível de 5\% há evidência suficiente de que o tempo médio de ativação excede 25s. Assim, rejeitamos $H_0$.

Para o nível de significância $\alpha = 0.01$, temos que $0.043 > 0.01$, ou seja, ao nível de 1\% não há evidência suficiente para afirmar que o tempo médio de ativação seja maior que 25s. Logo, não rejeitamos $H_0$.

\quest{69}

\textbf{Dados:}
\begin{itemize}
    \item Hipótese nula (\(H_0\)): \(\mu \geq 10\) psi
    \item Hipótese alternativa (\(H_1\)): \(\mu < 10\) psi
    \item Nível de significância (\(\alpha\)): 0,01
    \item Tamanho da amostra (\(n\)): 10
    \item Distribuição: Normal com \(\sigma\) desconhecido (teste \(t\))
    \item Desvio padrão real (\(\sigma\)): 0,80 psi (usado para cálculo do poder)
\end{itemize}

\subsection*{a. Probabilidade de aceitar \(H_0\) falsamente (Erro Tipo II)}

Queremos calcular \(\beta\) para \(\mu_1 = 9,5\) psi e \(\mu_2 = 9,0\) psi.

\textbf{Fórmula do poder do teste unilateral à esquerda:}
\[
\beta = P\Bigl( t_{n-1} \geq t_{\alpha, n-1} - \frac{|\mu_0 - \mu_1|}{\sigma/\sqrt{n}} \Bigr)
\]

\textbf{Passo 1: Valor crítico}

Para \(\alpha=0{,}01\) e \(n-1=9\) graus de liberdade:
\[
t_{0,01,9} \approx -2,821.
\]

\textbf{Passo 2: Cálculo para \(\mu_1=9,5\) psi}

\[
\Delta = \frac{|10 - 9,5|}{0,8/\sqrt{10}} 
= \frac{0,5}{0,253} \approx 1,976.
\]

\[
\beta = P\bigl(t_{9} \geq -2,821 + 1,976\bigr)
= P(t_{9} \geq -0,845) \approx 0,6.
\]

\textbf{Passo 3: Cálculo para \(\mu_1=9,0\) psi}

\[
\Delta = \frac{|10 - 9,0|}{0,8/\sqrt{10}}
= \frac{1,0}{0,253} \approx 3,953.
\]

\[
\beta = P\bigl(t_{9} \geq -2,821 + 3,953\bigr)
= P(t_{9} \geq 1,132) \approx 0,2.
\]

\textbf{Resposta:}

\[
\beta \approx 0,6 \text{ para } \mu=9,5 \text{ psi}; 
\quad \beta \approx 0,2 \text{ para } \mu=9,0 \text{ psi}.
\]

\subsection*{b. Tamanho da amostra para obter poder de 75\% (ou \(\beta=0,25\))}

\textbf{Fórmula para teste unilateral:}
\[
n \approx \biggl(\frac{(z_\alpha + z_\beta) \cdot \sigma}{\mu_0 - \mu_1}\biggr)^2.
\]

\textbf{Passo 1: Valores críticos}

\[
z_{0,01} = -2,326 
\quad\text{e}\quad
z_{0,25} \approx 0,674.
\]

\textbf{Passo 2: Cálculo de \(n\)}

\[
n \approx \biggl(\frac{(2,326+0,674) \cdot 0,8}{10 - 9,5}\biggr)^2 
= \biggl(\frac{3 \cdot 0,8}{0,5}\biggr)^2
= \biggl(\frac{2,4}{0,5}\biggr)^2
= 23,04.
\]

Arredondando para cima:
\[
n=24.
\]

\textbf{Observação:}

A discrepância com a resposta do livro (\(n=28\)) ocorre porque o livro usa a tabela \(t\), que exige \(n \geq 28\) para alcançar o poder exato. Usando software estatístico, chega-se a \(n=28\).

\quest{70}

Dados: $n = 20$, $\bar{x} = 9{,}872$ s, $s = 0{,}10$ s, $\mu_0 = 9{,}75$ s, $\alpha = 0{,}05$

Hip\'oteses:
\begin{itemize}
  \item $H_0: \mu \leq 9{,}75$
  \item $H_1: \mu > 9{,}75$
\end{itemize}

Como $n = 20$ e os dados s\~ao aproximadamente normais, usamos o teste $t$:
\[
t = \frac{\bar{x} - \mu_0}{s / \sqrt{n}} = \frac{9{,}872 - 9{,}75}{0{,}10 / \sqrt{20}} \approx 5{,}45
\]

Valor cr\'itico: $t_{0{,}05; 19} \approx 1{,}729$

Decis\~ao: $t = 5{,}45 > 1{,}729$, ent\~ao rejeitamos $H_0$.

Resposta final:
\[
\boxed{\text{Sim, h\'a evid\^encia de que o tempo m\'edio excede 9{,}75 s ($p < 0{,}0001$)}}
\] 

\quest{71}

\textbf{a)}

Dados: $n = 800$, $x = 16$, $\hat{p} = 0{,}02$, $p_0 = \frac{1}{75} \approx 0{,}0133$, $\alpha = 0{,}05$

Hipóteses:
\begin{itemize}
  \item $H_0: p = 0{,}0133$
  \item $H_1: p \neq 0{,}0133$
\end{itemize}

Verificamos que $np_0 \approx 10{,}64$ e $n(1 - p_0) \approx 789{,}36$, então podemos usar aproximação normal.

Estatística do teste:
\[
Z = \frac{\hat{p} - p_0}{\sqrt{\frac{p_0(1 - p_0)}{n}}} \approx \frac{0{,}02 - 0{,}0133}{0{,}00405} \approx 1{,}64
\]

Valor crítico: $z_{0{,}025} = \pm 1{,}96$

Decisão: Como $|Z| = 1{,}64 < 1{,}96$, não rejeitamos $H_0$.

Resposta final:
\[
\boxed{\text{a. } z = 1{,}64 < 1{,}96, \text{ portanto não rejeitamos } H_0. \text{ Tipo II}}
\]

\textbf{b)}

Valor $p$ aproximado: $2 \times P(Z > 1{,}64) \approx 0{,}10$

Como $p = 0{,}10 < 0{,}20$, rejeitamos $H_0$ para $\alpha = 0{,}20$.

Resposta final:
\[
\boxed{\text{b. Valor } p \approx 0{,}10. \text{ Para } \alpha = 0{,}20, \text{ rejeitamos } H_0. \text{ Sim}}
\]

\quest{72}

\textbf{Dados:}
\begin{itemize}
    \item Tamanho da amostra (\(n\)): 26
    \item Média amostral (\(\bar{x}\)): 1,89 mg
    \item Desvio padrão amostral (\(s\)): 0,42 mg
    \item Média populacional de referência (\(\mu_0\)): 1,75 mg/g
    \item Nível de significância (não especificado; adotamos \(\alpha=0,05\))
\end{itemize}

\subsection*{1. Hipóteses}

Queremos verificar se os novos dados contradizem o valor anterior (\(\mu_0=1,75\) mg/g).  
Como não foi indicada direção específica, utilizamos um teste bilateral:

\[
H_0: \mu = 1,75 
\quad\text{vs}\quad 
H_1: \mu \neq 1,75.
\]

\subsection*{2. Estatística de teste}

Como \(\sigma\) é desconhecido e \(n=26\), usamos o teste \(t\):

\[
t = \frac{\bar{x} - \mu_0}{s/\sqrt{n}}
= \frac{1,89 - 1,75}{0,42/\sqrt{26}}
= \frac{0,14}{0,0824} 
\approx 1,70.
\]

\subsection*{3. Valor P (teste bilateral)}

Com 25 graus de liberdade:

\[
P = 2 \cdot P(t_{25} > 1,70).
\]

Pela tabela \(t\):

\[
P(t_{25} > 1,70) \approx 0,05.
\]

\[
P \approx 2 \times 0,05 = 0,10.
\]

\subsection*{4. Conclusão}

Como \(P \approx 0,10 > \alpha=0,05\), \textbf{não rejeitamos \(H_0\)}.

Não há evidências estatísticas suficientes para concluir que a quantidade média de antitoxina necessária difere de 1,75 mg/g.

\subsection*{Validade das Suposições}

A análise é válida se:
\begin{itemize}
    \item A amostra for aleatória e representativa.
    \item A distribuição da quantidade de neutralizante for aproximadamente normal (importante especialmente para \(n\) pequeno); aqui \(n=26\) e a Teoria Central do Limite garante robustez.
    \item O desvio padrão amostral (\(s=0,42\)) for uma boa estimativa de \(\sigma\).
\end{itemize}

\textbf{Resposta final:}

Os dados não contradizem significativamente o valor anterior de 1,75 mg/g (\(P \approx 0,10\)).  
A análise é considerada válida mesmo sem normalidade estrita da população, dado o tamanho da amostra.


\quest{73}

Dados: $n = 45$, $\bar{x} = 3107$ psi, $s = 188$ psi, $\mu_0 = 3200$ psi, $\alpha = 0{,}001$

Hipóteses:
\begin{itemize}
  \item $H_0: \mu = 3200$
  \item $H_1: \mu < 3200$
\end{itemize}

Como $n = 45$, usamos o teste $t$:
\[
t = \frac{\bar{x} - \mu_0}{s / \sqrt{n}} = \frac{3107 - 3200}{188 / \sqrt{45}} \approx \frac{-93}{28{,}03} \approx -3{,}32
\]

Valor crítico: $t_{0{,}001; 44} \approx -3{,}26$

Decisão: $t = -3{,}32 < -3{,}26$, então rejeitamos $H_0$.

Resposta final:
\[
\boxed{\text{Sim, h\'a forte evid\^encia de que a m\'edia < 3200 psi ($p \approx 0{,}0007$)}}
\]

\quest{74}

Dados: $n = 72$, $x = 42$, $\hat{p} = \frac{42}{72} = 0{,}5833$, $p_0 = 0{,}75$

Hipóteses:
\begin{itemize}
  \item $H_0: p = 0{,}75$
  \item $H_1: p < 0{,}75$
\end{itemize}

Verificação das condições:
\begin{itemize}
  \item $np_0 = 72 \times 0{,}75 = 54 \geq 10$
  \item $n(1 - p_0) = 72 \times 0{,}25 = 18 \geq 10$
\end{itemize}

Estatística do teste:
\[
Z = \frac{\hat{p} - p_0}{\sqrt{\frac{p_0(1 - p_0)}{n}}} = \frac{0{,}5833 - 0{,}75}{\sqrt{\frac{0{,}75 \times 0{,}25}{72}}} = \frac{-0{,}1667}{0{,}0510} \approx -3{,}27
\]

Valor $p$ (unicaudal):
\[
P(Z \leq -3{,}27) \approx 0{,}00054
\]

Decisão: Como $p < 0{,}05$, rejeitamos $H_0$.

Resposta final:
\[
\boxed{\text{Sim, h\'a forte evid\^encia de que a propor\c{c}\~ao < 0{,}75 ($p \approx 0{,}00054$)}}
\]

\quest{75}

Seja \( X_1, X_2, \ldots, X_n \) uma amostra de variáveis independentes com distribuição de Poisson de parâmetro \( \lambda \). Deseja-se testar:

\[
H_0: \lambda = 4,0 \quad \text{vs} \quad H_1: \lambda > 4,0
\]

Com base em:
\[
n = 36,\quad \sum X_i = 160 \Rightarrow \bar{X} = \frac{160}{36} \approx 4{,}44
\]

Sabemos que, para \( n \) grande, a estatística:
\[
Z = \frac{\bar{X} - \lambda_0}{\sqrt{\lambda_0/n}} \sim \mathcal{N}(0,1)
\]

Substituindo os valores:
\[
Z = \frac{4{,}44 - 4{,}0}{\sqrt{4{,}0 / 36}} = \frac{0{,}44}{0{,}3333} \approx 1{,}33
\]

Como o teste é unilateral à direita com \( \alpha = 0{,}02 \), o valor crítico é:
\[
z_{0{,}98} \approx 2{,}05
\]

Como \( Z = 1{,}33 < 2{,}05 \), não rejeitamos \( H_0 \).

\[
\boxed{\text{Não, visto que } z = 1{,}33 < 2{,}05.}
\]

\quest{76}

Dados: $n = 32$, $\bar{x} = 17{,}5$ min, $s = 2{,}2$ min, $\mu_0 = 15$ min, $\alpha = 0{,}05$

Hip\'oteses:
\begin{itemize}
  \item $H_0: \mu \leq 15$
  \item $H_1: \mu > 15$
\end{itemize}

Como a distribui\c{c}\~ao \'{e} aproximadamente normal, usamos a estat\'istica:
\[
t = \frac{\bar{x} - \mu_0}{s / \sqrt{n}} = \frac{17{,}5 - 15}{2{,}2 / \sqrt{32}} \approx \frac{2{,}5}{0{,}3889} \approx 6{,}43
\]

Valor $P$: Para $t = 6{,}43$ e $gl = 31$, $P < 0{,}0001$

Decis\~ao: Como $P < 0{,}05$, rejeitamos $H_0$.

Resposta final:
\[
\boxed{\text{Sim, os dados deixam d\'uvida na alega\c{c}\~ao da empresa, pois o valor $P$ \'{e} menor que 0{,}05}}
\]

\textbf{77.} Para testar $H_0: \sigma \leq 0{,}50$ contra $H_a: \sigma > 0{,}50$, com $n = 10$ e $s = 0{,}58$, utilizamos a estatística qui-quadrado:
\[
\chi^2 = \frac{(n - 1) s^2}{\sigma_0^2} = \frac{(10 - 1)(0{,}58)^2}{0{,}50^2} = \frac{9 \cdot 0{,}3364}{0{,}25} = \frac{3{,}0276}{0{,}25} = 12{,}11
\]

O valor crítico da distribuição qui-quadrado com $n - 1 = 9$ graus de liberdade e $\alpha = 0{,}01$ (tabela A.8) é:
\[
\chi^2_{0{,}01;9} = 21{,}665
\]

Como $12{,}11 < 21{,}665$, não rejeitamos $H_0$.

\textbf{Conclusão:} Não, visto que $12{,}11 < 21{,}665$. Portanto, não há evidências suficientes para concluir que o desvio padrão real excede $0{,}50^\circ$C ao nível de significância de 1\%.

\quest{78}
Como hipóteses temos $H_0: \sigma^2 = 0.04$ \textit{versus} $H_a: \sigma^2 < 0.04$. Como $n=21$, então $gl = n-1 = 20$. A estatística de teste é dada por: $\chi^2 = (n – 1)·s^2/σ_0^2 = 20·s^2/0.04 = 8.58$.\\Rejeitamos $H_0$ se $\chi^2 \le \chi^2_{\alpha, 20}$, o enunciado diz que $\alpha = 0.01$. Pela tabela, obtemos que $\chi^2_{20,0.99} = 8.260$. Note que usamos 0.99 ($1-0.01$) pois queremos o quantil inferior. Assim $P(\chi^2 \le 8.260) = 0.01$.\\ $\chi^2_{20,0.975} = 9.591 \Rightarrow P(\chi^2 \le 9.591) = 0.025$. Portanto, obtemos $8,260 < 8,58 < 9,591 \Rightarrow 0,01 < P(\chi^2 \le 8,58) < 0,025 \Rightarrow 0,01 < \text{valor-p} < 0,025$.\\
A região crítica para $\alpha = 0,01$ (cauda à esquerda): $\chi \le \chi^2_{20,0.99} = 8,260$.
Como $8,58 > 8,260$, não rejeitamos $H_0$ ao nível de 1\%. Conclui-se que já que o valor-p está entre 0.01 e 0.025, então Em $\alpha = 0,01$ não há evidência suficiente para concluir que $\sigma^2 < 0,04$.

\quest{79}
\begin{enumerate}
    \item Estimador aproxim. não-tendencioso\\
    Como $E(\overline X)=\mu$ e $E(S)\approx\sigma$, um estimador natural é
    $\widehat\theta =\overline X \;+\;2{,}33\,S.$
    De fato, $E(\widehat\theta) =E(\overline X)+2{,}33\,E(S) \approx\mu+2{,}33\,\sigma =\theta.$

    \item Variância e erro-padrão
    
    Pela independência,
    \begin{align*}
    \widehat\theta &=\overline X+(2{,}33)^2\,S =\frac{\sigma^2}{n}+(2{,}33)^2\;\frac{\sigma^2}{2n} \\ &=\frac{\sigma^2}{n}\Bigl[\,1+\tfrac{(2{,}33)^2}{2}\Bigr]. \end{align*}
    Logo o desvio-padrão de $\widehat\theta$ é
    $$\sigma_{\widehat\theta} =\frac{\sigma}{\sqrt n}\, \sqrt{\,1+\frac{(2{,}33)^2}{2}\,} \approx\frac{\sigma}{\sqrt n}\;1{,}927.$$
    Substituindo $\sigma$ por $s$, o erro-padrão estimado é
    $$\widehat\sigma_{\widehat\theta} =\frac{s}{\sqrt n}\, \sqrt{\,1+\frac{(2{,}33)^2}{2}\,}.$$

    \item Teste de $H_0:\theta=\theta_0$ versus $H_a:\theta<\theta_0$\\
    Como $\widehat\theta\approx N(\theta,\sigma_{\widehat\theta}^2)$, sob $H_0$ a estatística
    $Z =\frac{\widehat\theta-\theta_0}{\widehat\sigma_{\widehat\theta}} \;\dot{\sim}\;N(0,1).$
    
    No problema do pH do solo temos: $n=64, \overline x=6,33 \text{ e } s=0,16.$ Então:
    $$\widehat\theta =6{,}33+2{,}33\times0{,}16 =6{,}7028,$$
    $$\widehat\sigma_{\widehat\theta} =\frac{0{,}16}{\sqrt{64}} \sqrt{1+\frac{(2{,}33)^2}{2}} =0{,}02\times1{,}927 \approx0{,}0385.$$
    Testamos $H_0:\theta=6{,}75 \quad\text{versus}\quad H_a:\theta<6{,}75,$
    ao nível $\alpha=0,01$. A estatística calculada é $Z =\frac{6{,}7028 - 6{,}75}{0{,}0385} \approx -1{,}23.$
    O $p$-valor unilateral à esquerda é $p=\Phi(-1{,}23)\approx0{,}11.$
    Como $p=0{,}11>0{,}01$, \emph{não rejeitamos} $H_0$.  
    Não há evidência significativa, a 1\% de nível, de que mais de 99\% das amostras de solo tenham pH abaixo de 6,75.
\end{enumerate}

\quest{80}
Dados: amostra exponencial $X_1, X_2, \dots, X_{10}$ com parâmetro $\lambda$, sendo que $2\lambda \sum X_i \sim \chi^2_{2n}$. Hipótese nula: $H_0: \mu = \mu_0$, equivalente a $\lambda = \lambda_0 = 1/\mu_0$. Dados observados: 95, 16, 11, 3, 42, 71, 225, 64, 87, 123; $n=10$ e $\mu_0=75$.\\

\textbf{a.} Para testar $H_0: \mu=\mu_0$ contra alternativas:
\begin{itemize}
    \item Unilateral à direita ($H_1: \mu > \mu_0$): rejeitamos $H_0$ se $T> \chi^2_{2n,1-\alpha}$.
    \item Unilateral à esquerda ($H_1: \mu < \mu_0$): rejeitamos $H_0$ se $T< \chi^2_{2n,\alpha}$.
    \item Bicaudal ($H_1: \mu \neq \mu_0$): rejeitamos $H_0$ se $T< \chi^2_{2n,\alpha/2}$ ou $T> \chi^2_{2n,1-\alpha/2}$.
\end{itemize}

\textbf{b.} Calculando:
\[
\sum X_i=95+16+11+3+42+71+225+64+87+123=737.
\]
\[
\lambda_0=1/75, \quad T=2\cdot \frac{1}{75}\cdot 737 \approx 19,653.
\]
Com $2n=20$, para $\alpha=0,05$, temos $\chi^2_{20,0.05} \approx 10,851$. Como $19,653>10,851$, não rejeitamos $H_0$.\\
Valor-p: $P(\chi^2_{20}\le 19,653)\approx 0,48$.\\

\textbf{Conclusão:} $P>0,05$, então não há evidências estatísticas para concluir que $\mu<75$ horas.

\quest{81}
\begin{enumerate}
    \item Provar que $\mathrm{P}(\text{erro tipo I})=\alpha$.
    \begin{align*} \mathrm{P}(\text{rejeitar }H_0) &= \mathrm{P}\bigl(Z \le -\,z_{\alpha-\gamma}\bigr) + \mathrm{P}\bigl(Z \ge z_{\gamma}\bigr) \\ &= \Phi\bigl(-\,z_{\alpha-\gamma}\bigr) + \bigl[1 - \Phi(z_{\gamma})\bigr] \\ &= \bigl[1 - \Phi(z_{\alpha-\gamma})\bigr] + \gamma \quad(\text{pois }1-\Phi(z_{\gamma})=\gamma) \\ &= (\alpha-\gamma)+\gamma =\alpha.
    \end{align*}
    
    \item Expressão para $\beta(\mu')$.
    Sob $H\colon\mu=\mu'$, a estatística satisfaz
    $$Z\;\sim\;N\!\Bigl(\delta,1\Bigr), \quad \delta = \frac{\mu' - \mu_0}{\sigma/\sqrt n}.$$
    O teste \emph{aceita} $H_0$ quando
    $-\,z_{\alpha-\gamma} < Z < z_{\gamma}$.  
    Logo a probabilidade de erro–II é
    $$\begin{aligned} \beta(\mu') &= \mathrm{P}\bigl(-z_{\alpha-\gamma} < Z < z_{\gamma}\bigr) \\ &= \Phi\bigl(z_{\gamma}-\delta\bigr) - \Phi\bigl(-\,z_{\alpha-\gamma}-\delta\bigr). \end{aligned}$$
    
    \item Comparação de $\beta(\mu_0+\Delta)$ e $\beta(\mu_0-\Delta)$.
    Seja $\Delta>0$ e defina
    $$d \;=\;\frac{\Delta}{\sigma/\sqrt n}>0.$$
    Então
    $$\begin{cases} \displaystyle \beta(\mu_0+\Delta) =\Phi\bigl(z_{\gamma}-d\bigr) -\Phi\bigl(-z_{\alpha-\gamma}-d\bigr), \\[1em] \displaystyle \beta(\mu_0-\Delta) =\Phi\bigl(z_{\gamma}+d\bigr) -\Phi\bigl(-z_{\alpha-\gamma}+d\bigr). \end{cases}$$
    Mostra-se que
    $$\beta(\mu_0+\Delta)\;<\;\beta(\mu_0-\Delta) \quad\Longleftrightarrow\quad z_{\gamma} < z_{\alpha-\gamma} \quad\Longleftrightarrow\quad \gamma > \frac\alpha2.$$
\end{enumerate}

\quest{82}
Seja $p = P(\text{aprovação})$. Temos as hipóteses $H_0: p=0,90$ \textit{versus} $H_a: p \neq 0,90$. Suponha que $n=2$ pessoas façam o exame e $X$ seja o número de aprovados. Então $X \sim Binomial(n=2, p)$.\\

\textbf{a.} A região de cauda inferior $\{0,1,5\}$ indica que $X \in \{0,1\}$ no lado inferior, mas como $n=2$, só existem valores possíveis $\{0,1,2\}$. Então, a região de cauda inferior $\{0,1\}$ teria:
\[
P(X=0 \text{ ou } 1 \mid p=0,90) = (1-p)^2 + 2p(1-p) = (0,10)^2 + 2 \cdot 0,90 \cdot 0,10 = 0,01+0,18=0,19.
\]
Portanto, especificar $\{0,1\}$ como região crítica inferior daria nível de significância de $0,19$, que é muito maior que $0,017$.\\

\textbf{b.} Como $H_a$ é bicaudal, deveríamos dividir $\alpha=0,01$ em duas caudas de $0,005$ cada. Mas não há valores inteiros de $X$ para montar uma região crítica que tenha probabilidade total de $0,01$, pois $P(X=0)=0,01$ e $P(X=2)=0,81$. Se incluirmos $X=0$ e $X=2$, o nível passa a $0,01+0,81=0,82$. Assim, nenhum teste bicaudal exato atinge nível $0,01$.\\

\textbf{c.} Vamos construir a função poder $\beta(p')=P(\text{rejeitar }H_0 \mid p')$. Se usarmos a região crítica $\{0,2\}$, então:
\[
\beta(p')=P(X=0 \mid p')+P(X=2 \mid p') = (1-p')^2 + (p')^2.
\]
O gráfico de $\beta(p')$ será mínimo em $p'=0,5$ (valor mais difícil de detectar diferença) e máximo em $p'=0$ ou $p'=1$.\\
Não é desejável pois o teste fica com alto nível de significância ou poder muito baixo perto de $p'=0,90$.\\

\end{document}
