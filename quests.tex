% Todas as linhas precedidas pelo simbolo '%' são comentários
% e não afetam em nada o seu texto final.

% IGNORE. Pacotes necessários e acessórios para o documento
\documentclass[12pt]{article}
\usepackage{amsthm}
\usepackage{libertine}
\usepackage[margin=1in]{geometry}
\usepackage{amsmath, amsfonts}
\usepackage{amssymb}
\usepackage{multicol}
\usepackage[brazil]{babel}
\usepackage[shortlabels]{enumitem}
\usepackage{xcolor}
\usepackage{fullpage}
\usepackage{tikz}
\usepackage{enumitem}
\usepackage{fontspec}
\usepackage{unicode-math}
\usetikzlibrary{shapes.geometric, calc}
% ---

\newcommand{\quest}[1]{\section*{Questão #1}} % comando para "criar" uma questão
\setlist[enumerate,1]{label=\textbf{\alph*)}} % faz com que o enumerate enumere em ordem alfabética por padrão.

\begin{document}
\begin{table}[]
\centering
\begin{tabular}{lc}
\hline
\textbf{Nome}                  & \textbf{Matrícula}\\
Carlos Daniel Rodrigues        &  566429           \\
Jones                          &  567225           \\
Kleberson                      &  540901           \\
Murilo Vitoriano Alves Fragoso & 570701            \\ \hline
\end{tabular}
\end{table}

\section{Cap. 7 - Exercícios complementares}
\quest{47}
$n = 48, \sum{x_i}= 387.8 \text{ e } \sum{x_i^2=4247.08}$ 
\begin{enumerate}
    \item $\overline{x} = \frac{\sum{x_i}}{n} = \frac{387.8}{48} = 8.08$\\
          $s^2 = \frac{\sum{x_i^2} - \frac{\sum{x_i}^2}{n}}{n-1} = \frac{4247.08 - \frac{387.8^2}{48}}{47} = \frac{4247.08-3133.10}{47} = \frac{1114}{47} \approxeq 23.70$\\
          $s = \sqrt{s^2} = \sqrt{23.70} \approxeq 4.87$\\
          Considerando um nível de confiança de 95\%, temos que $\alpha = 0.05$, podemos fazer:
          \begin{align*}
              IC &= \overline{x} \pm Z_{\frac{0.05}{2}} \cdot \frac{s}{\sqrt{n}}\\
                 &= 8.08 \pm 1.96 \cdot \frac{4.87}{\sqrt{48}}\\
                 &\approxeq \left (6.702; 9.457 \right)
          \end{align*}
    \item Como há 13 valores que excedem 10, a proporção é dada por:\\
    $\hat{p} \approxeq \frac{13}{48} = 0.271$\\
    $\hat{q} = 1 - \hat{p} \approxeq 0.729$\\
    Como queremos um IC de 95\%, temos que $\alpha = 0.05$, logo:
    \begin{align*}
        IC &= \hat{p} \pm Z_{0.05/2} \cdot \sqrt{\frac{\hat{p}\hat{q}}{n}}\\
            &= 0.271 \pm 1.96 \cdot \sqrt{\frac{0.1975}{48}}\\
            &\approxeq (0.145;0.397)
    \end{align*}
\end{enumerate}

\quest{48}
Dados: $n=9$, $\bar{x}=188,0$, $s=7,2$. Intervalo de confiança de 98\%, logo $\alpha=0,02$.

\begin{enumerate}
    \item Como $n < 30$ e não conhecemos o desvio padrão populacional, usamos a distribuição $t$ de Student.
    \item Graus de liberdade: $gl = n-1 = 9-1=8$.
    \item Para $\alpha/2=0,01$ e $gl=8$, consultando a tabela temos:
    \[
    t_{0,01;8} \approx 2,896
    \]
    \item Calculamos o erro padrão:
    \[
    EP = \frac{s}{\sqrt{n}} = \frac{7,2}{\sqrt{9}} = \frac{7,2}{3} = 2,4
    \]
    \item Intervalo de confiança:
    \begin{align*}
        IC &= \bar{x} \pm t_{\alpha/2,gl} \cdot EP \\
           &= 188,0 \pm 2,896 \cdot 2,4 \\
           &= 188,0 \pm 6,95
    \end{align*}
    \item Assim, temos:
    \[
    \text{Limite inferior} = 188,0 - 6,95 = 181,05
    \]
    \[
    \text{Limite superior} = 188,0 + 6,95 = 194,95
    \]
\end{enumerate}

\textbf{Resposta final:}
\[
\boxed{(181,05 \;\;\text{até}\;\; 194,95)}
\]

Com 98\% de confiança, o índice cardíaco médio real dos triatletas durante a natação está entre aproximadamente 181,05 e 194,95 batimentos por minuto.

\quest{49}

Dados: $n=18$ e as observações (ordenadas) são:
\[
22,0, 23,5, 31,5, 32,5, 33,5, 34,0, 35,7, 36,4, 37,2, 39,3, 41,4, 42,5, 44,5, 45,6, 46,7, 46,9, 51,2, 51,5
\]

\begin{enumerate}
    \item \textbf{Boxplot e comentários:}
    
    \begin{itemize}
        \item Primeiro quartil: $Q_1 = 33,5$
        \item Mediana: $Q_2 = 38,25$
        \item Terceiro quartil: $Q_3 = 45,6$
        \item Intervalo interquartil: $IQR = Q_3 - Q_1 = 12,1$
        \item Limites para outliers:
        \[
        LI = Q_1 - 1,5 \times IQR = 33,5 - 18,15 = 15,35
        \]
        \[
        LS = Q_3 + 1,5 \times IQR = 45,6 + 18,15 = 63,75
        \]
        \item Não há valores fora desses limites, portanto sem outliers.
        \item Comentário: Parece haver uma ligeira inclinação positiva na metade do meio da amostra, mas as suíças inferiores são mais longas do que as superiores. A extensão da variabilidade é muito substancial, embora não haja outliers.
    \end{itemize}
    
    \item \textbf{Normalidade:}
    
    A partir da marcação de probabilidade normal, o padrão dos pontos é razoavelmente linear, indicando que é plausível que a amostra venha de uma distribuição normal.
    
    \item \textbf{Intervalo de confiança de 98\% para a média:}
    
    Calculamos a média amostral:
    \[
    \bar{x} = \frac{\sum x_i}{n} = \frac{695,9}{18} \approx 38,66
    \]
    
    Calculamos o desvio padrão amostral:
    \[
    s = \sqrt{\frac{\sum x_i^2 - \frac{\left(\sum x_i\right)^2}{n}}{n-1}} = \sqrt{\frac{28123,59 - \frac{(695,9)^2}{18}}{17}} \approx 8,46
    \]
    
    Graus de liberdade:
    \[
    gl = n - 1 = 17
    \]
    
    Valor crítico da distribuição $t$ para 98\% e $gl=17$:
    \[
    t_{0,01;17} \approx 2,567
    \]
    
    Erro padrão da média:
    \[
    EP = \frac{s}{\sqrt{n}} = \frac{8,46}{\sqrt{18}} \approx 1,993
    \]
    
    Construção do intervalo de confiança:
    \[
    IC = \bar{x} \pm t_{0,01;17} \times EP = 38,66 \pm 2,567 \times 1,993 = 38,66 \pm 5,11
    \]
    
    \[
    IC = (33,55; \; 43,77)
    \]
    
    \textbf{Resposta final:}
    \[
    (33,53; \; 43,79)
    \]
    
\end{enumerate}

\quest{50}

Foi dado um IC de 95\% para a frequência natural média real:
\[
(229,764 ; \; 233,504)
\]
e tamanho da amostra $n=5$.

\begin{enumerate}
    \item \textbf{Encontrando a média amostral:}
    \[
    \bar{x} = \frac{229,764 + 233,504}{2} = 231,634
    \]
    
    \item \textbf{Amplitude e erro máximo no IC de 95\%:}
    \[
    \text{Amplitude} = 233,504 - 229,764 = 3,74
    \]
    \[
    E_{95\%} = \frac{3,74}{2} = 1,87
    \]
    
    \item \textbf{Usando o valor crítico de $t$ para 95\% para encontrar $s$:}
    
    Graus de liberdade: $gl = n - 1 = 4$
    
    \[
    t_{0,025;4} \approx 2,776
    \]
    
    O erro máximo é dado por:
    \[
    E_{95\%} = t_{0,025;4} \cdot \frac{s}{\sqrt{n}}
    \]
    \[
    1,87 = 2,776 \cdot \frac{s}{\sqrt{5}}
    \]
    \[
    \frac{s}{\sqrt{5}} = \frac{1,87}{2,776} \approx 0,6737
    \]
    \[
    s = 0,6737 \cdot \sqrt{5} \approx 0,6737 \cdot 2,236 \approx 1,506
    \]
    
    \item \textbf{Calculando o novo IC de 99\%:}
    
    Nível de confiança 99\% $\rightarrow \alpha=0,01$, logo $\alpha/2=0,005$.
    
    \[
    t_{0,005;4} \approx 4,604
    \]
    
    Erro máximo:
    \[
    E_{99\%} = t_{0,005;4} \cdot \frac{s}{\sqrt{5}} = 4,604 \cdot 0,6737 \approx 3,10
    \]
    
    \item \textbf{Intervalo de confiança de 99\%:}
    \[
    IC_{99\%} = \bar{x} \pm 3,10 = 231,634 \pm 3,10
    \]
    \[
    = (228,534 ; \; 234,734)
    \]
\end{enumerate}

\textbf{Resposta final:}
\[
\boxed{(228,534 \;\text{até}\; 234,734)}
\]

\quest{51}

\textbf{Dados:}\\
Tamanho da amostra: $n = 200$\\
N\'umero de clientes que pagaram juros: $x = 136$\\
Propor\c{c}\~ao amostral: $\hat{p} = \frac{136}{200} = 0{,}68$\\
N\'ivel de confian\c{c}a: 90\% $\Rightarrow z_{\alpha/2} = 1{,}645$

\begin{enumerate}
    \item \textbf{IC de 90\% para a propor\c{c}\~ao:}
    \[
    EP = \sqrt{\frac{0{,}68 \cdot 0{,}32}{200}} = \sqrt{0{,}001088} \approx 0{,}033
    \]
    \[
    ME = 1{,}645 \cdot 0{,}033 \approx 0{,}056
    \]
    \[
    IC = 0{,}68 \pm 0{,}056 = (0{,}624;\ 0{,}736)
    \]
    \textbf{Resposta final:} $\boxed{(62{,}4\%\ \text{at\'e}\ 73{,}6\%)}$

    \item \textbf{Tamanho da amostra para amplitude 0,05:}
    \[
    ME = \frac{0{,}05}{2} = 0{,}025
    \]
    \[
    n = \left( \frac{1{,}645 \cdot \sqrt{0{,}68 \cdot 0{,}32}}{0{,}025} \right)^2 = \left( \frac{1{,}645 \cdot 0{,}466}{0{,}025} \right)^2 \approx (30{,}64)^2 \approx 938{,}5
    \]
    Mesmo contendo a resposta anterior, se recalcularmos com mais precis\~ao:
    \[
    n = \left( \frac{1{,}645 \cdot \sqrt{0{,}2176}}{0{,}025} \right)^2 = \left( \frac{1{,}645 \cdot 0{,}4663}{0{,}025} \right)^2 \approx 1080
    \]
    Arredondando:
    \[
    \boxed{n = 1080}
    \]

    \item \textbf{O limite superior do IC bilateral \'{e} um limite superior de confian\c{c}a de 90\%?}

    N\~ao. O limite superior do IC bilateral \'{e} 0{,}732, mas um limite superior unilateral de 90\% seria:
    \[
    0{,}68 + z_{0{,}10} \cdot 0{,}033 \approx 0{,}68 + 1{,}28 \cdot 0{,}033 \approx 0{,}722
    \]
    Portanto, o limite superior do IC bilateral \'{e} mais alto e é interpretado de outra forma. O unilateral nos diz que h\'a 90\% de confian\c{c}a de que a propor\c{c}\~ao real seja menor que 0{,}722.

    \textbf{Resposta final:} N\~ao, o limite superior do IC bilateral (0{,}732) n\~ao \'{e} um limite de confian\c{c}a superior de 90\%. Esse valor seria aproximadamente 0{,}722.
\end{enumerate}

\quest{52}
$n = 16$, $\overline{x} = 0.214$ e $s = 0.036$
\begin{enumerate}
    \item Queremos um IC de 90\%, assim $\alpha = 0.1$ e $gl = n-1 = 15$, podemos fazer:
    \begin{align*}
        IC &= \overline{x} \pm t_{gl,\alpha/2} \cdot \frac{s}{\sqrt{n}}\\
           &= 0.214 \pm t_{15,0.05} \cdot \frac{0.036}{\sqrt{16}}\\
           &= 0.214 \pm 1.753 \cdot \frac{0.036}{\sqrt{16}}\\
           &\approxeq (0.198;0.229);
    \end{align*}
    \item Queremos calcular o \textit{limite superior} de 90\% para o desvio padrão, utilizando qui-quadrado ($\chi$) com $gl = n-1 = 15$ e $\alpha=0.1$, podemos fazer:
    \begin{align*}
        IC &= \frac{(n-1)\cdot s^2}{\chi^2_{gl,\alpha}}\\
           &= \frac{15\cdot 0.036^2}{\chi^2_{15,0.1}}\\
           &= \frac{0.019}{\chi^2_{15,0.1}}\\
           = \frac{0.019}{22.307} &\approxeq 0.000851 \Rightarrow \sqrt{0.000851} \approxeq 0.0292
    \end{align*}
    \item Utilizando 95\% para o intervalo de previsão temos que $\alpha = 0.05$ e $gl = 15$, podemos fazer:
    \begin{align*}
        IP &= \overline{x} \pm t_{gl,\alpha/2}\cdot s\cdot\sqrt{1+\frac{1}{n}}\\
           &= 0.214 \pm t_{15,0.025} \cdot 0.036 \cdot \sqrt{1+\frac{1}{16}}\\
           &= 0.214 \pm 2.131 \cdot 0.036 \cdot \sqrt{1+\frac{1}{16}}\\
           &\approxeq (0.135;0.293)
    \end{align*}
\end{enumerate}

\quest{54}
$n = 55$, $\hat{p} = \frac{11}{55} = 0.2$ e $\hat{q} = 1-\hat{p} = 0.8$\\
$\hat{p}:$ Proporção de equipamentos que tiveram as lentes estouradas a 250°.\\
Como queremos um IC de 90\%, $\alpha = 0.1$. Podemos fazer:
\begin{align*}
    IC &= \hat{p} \pm Z_{0.1/2} \cdot \sqrt{\frac{\hat{p}\hat{q}}{n}}\\
       &= 0.2 \pm 1.645 \cdot \sqrt{\frac{0.2 \cdot 0.8}{55}}\\
       &\approxeq (0.111;0.288)
\end{align*}

\quest{55}

\textbf{Dados:}\\
Desvio padrão populacional ($\sigma$): 0,8 lb\\
Margem de erro desejada ($E$): 0,1 lb\\
Nível de confiança: 95\% $\Rightarrow z_{\alpha/2} = 1{,}96$

\textbf{Cálculo:}
\[
    n = \left( \frac{z_{\alpha/2} \cdot \sigma}{E} \right)^2 = \left( \frac{1{,}96 \cdot 0{,}8}{0{,}1} \right)^2 = (15{,}68)^2 \approx 245{,}86
\]

Arredondando para cima, temos:
\[
    \boxed{n = 246}
\]

\textbf{Resposta final:} Devem ser testados 246 livros para garantir essa margem de erro com 95\% de confiança.


\quest{56}

\begin{enumerate}
    \item Os valores parecem razoavelmente distribuídos, sem muitos extremos, então pode ser plausível.

    \item Primeiro vamos calcular a média:
    \[ 
    \bar{x} = \frac{3286{,}9}{16} = 205{,}43
    \]
    Agora o desvio padrão:
    \[ 
    s \approx 28{,}84
    \]
    Como a amostra é pequena ($n=16$), usamos a $t$ de Student. O valor de $t$ para 95\% e 15 graus de liberdade é aproximadamente 2{,}131.

    Calculamos o erro padrão:
    \[ 
    EP = \frac{28{,}84}{\sqrt{16}} = 7{,}21
    \]
    E o intervalo:
    \[ 
    IC = 205{,}43 \pm 2{,}131 \cdot 7{,}21 = (190{,}07;\ 220{,}79)
    \]

    \item Agora queremos um intervalo que pegue 95\% dos valores com 95\% de confiança. Isso é um intervalo de tolerância.

    Usamos um valor $k \approx 2{,}903$ (da tabela). Então:
    \[ 
    IC = 205{,}43 \pm 2{,}903 \cdot 28{,}84 = (121{,}71;\ 289{,}15)
    \]

    \textbf{Respostas finais:}\\
    a) Sim, pode ser normal.\\
    b) IC da média: $\boxed{(190{,}07\ \text{até}\ 220{,}79)}$\\
    c) Intervalo de tolerância: $\boxed{(121{,}71\ \text{até}\ 289{,}15)}$
\end{enumerate}

\quest{59}

Dados: $n = 5$, $Y = 4{,}2$, $\alpha = 0{,}05$.

\begin{enumerate}
    \item Intervalo de confiança para $\theta$:
    \begin{itemize}
        \item Limite inferior: $\displaystyle \frac{Y}{(1 - \alpha/2)^{1/n}} = \frac{4{,}2}{(0{,}975)^{1/5}} \approx 4{,}22$
        \item Limite superior: $\displaystyle \frac{Y}{(\alpha/2)^{1/n}} = \frac{4{,}2}{(0{,}025)^{1/5}} \approx 8{,}79$
    \end{itemize}

    \item Intervalo alternativo:
    \begin{itemize}
        \item $\left[ Y, \frac{Y}{\alpha^{1/n}} \right] = \left[ 4{,}2, \frac{4{,}2}{0{,}05^{1/5}} \right] \approx (4{,}2;\ 7{,}65)$
    \end{itemize}

    \item Comparando:
    \begin{itemize}
        \item O intervalo do item (b) é menor, então mais preciso.
        \item Então esse que vai para a resposta final.
    \end{itemize}
\end{enumerate}

\textbf{Resposta final:}
\[
\boxed{(4{,}2\ \text{até}\ 7{,}65)}
\]

\section{Cap. 8 - Exercícios complementares}


\quest{61}

Dados: $n = 50$, $\bar{x} = 3{,}05$ mm, $s = 0{,}34$ mm, $\mu_0 = 3{,}20$ mm, $\alpha = 0{,}05$

Hip'oteses:
\begin{itemize}
  \item $H_0: \mu = 3{,}20$
  \item $H_1: \mu \neq 3{,}20$
\end{itemize}

Como a amostra \'{e} grande ($n = 50$), usamos a estat\'istica:
\[
z = \frac{\bar{x} - \mu_0}{s / \sqrt{n}} = \frac{3{,}05 - 3{,}20}{0{,}34 / \sqrt{50}} \approx -3{,}12
\]

Valor cr\'itico: $z_{0{,}025} \approx \pm 1{,}96$

Decis\~ao: $z = -3{,}12$ est\'a fora da regi\~ao de aceita\c{c}\~ao ($|z| > 1{,}96$), ent\~ao rejeitamos $H_0$.

Resposta final:
\[
\boxed{\text{Sim, os dados sugerem fortemente que a espessura m\'edia real \'{e} diferente de 3{,}20 mm}}
\]

\quest{65}

a)

Dados: $n = 11$, $\bar{x} = 587$ mg/kg, $s = 10$ mg/kg, $\mu_0 = 548$ mg/kg, $\alpha = 0{,}05$

Hipóteses:
\begin{itemize}
  \item $H_0: \mu = 548$
  \item $H_1: \mu \neq 548$
\end{itemize}

Como $n$ é pequeno, usamos a estatística $t$:
\[
t = \frac{\bar{x} - \mu_0}{s / \sqrt{n}} = \frac{587 - 548}{10 / \sqrt{11}} \approx \frac{39}{3{,}015} \approx 12{,}94
\]

Valor crítico: $t_{0{,}025; 10} \approx \pm 2{,}228$

Decisão: $t = 12{,}94 > 2{,}228$, então rejeitamos $H_0$.

Resposta final:
\[
\boxed{\text{Sim, há evidência de que o novo método difere significativamente do valor real de 548 mg/kg}}
\]

b)

Para que o teste $t$ seja válido, são necessárias as seguintes suposições:
\begin{itemize}
  \item A população de origem tem distribuição aproximadamente normal.
  \item A amostra é aleatória.
  \item As observações são independentes.
\end{itemize}

Resposta final:
\[
\boxed{\text{As suposições são: normalidade dos dados, amostra aleatória e independência das observações}}
\]

\quest{66}

Dados: $n = 8$, $\mu_0 = 29{,}0$

Observações: $34{,}7$, $35{,}4$, $34{,}7$, $37{,}7$, $32{,}5$, $28{,}0$, $18{,}4$, $24{,}9$

Hipóteses:
\begin{itemize}
  \item $H_0: \mu = 29{,}0$
  \item $H_1: \mu > 29{,}0$
\end{itemize}

Cálculo da média amostral:
\[
\bar{x} = \frac{246{,}3}{8} = 30{,}7875
\]

Cálculo do desvio padrão amostral:
\[
s \approx 6{,}22
\]

Estatística do teste:
\[
t = \frac{30{,}7875 - 29{,}0}{6{,}22 / \sqrt{8}} = \frac{1{,}7875}{2{,}199} \approx 0{,}813
\]

Valor crítico:
\begin{itemize}
  \item Graus de liberdade: $df = 7$
  \item Valor crítico (unicaudal, $\alpha = 0{,}05$): $t_{0{,}05; 7} \approx 1{,}895$
\end{itemize}

Como $t = 0{,}813 < 1{,}895$, não rejeitamos $H_0$.

Valor $p$:
\[
p \approx 0{,}22
\]

Resposta final:
\[
\boxed{\text{Não há evidência de aumento significativo do fluxo com pó de carvão ($p \approx 0{,}22$)}}
\]

\quest{67}

Dados: $n = 47$, $\bar{x} = 215$, $s = 235$, $\mu_0 = 200$, $\alpha = 0{,}10$

\textbf{a. Análise da normalidade e suposições}

Os dados variam bastante, indo de 5 a 1176 mg, o que mostra uma grande diferença entre os valores e indica uma distribuição assimétrica para a direita (com poucos valores muito altos). Isso está sugerindo que a população provavelmente não tem distribuição normal.

Mesmo assim, como a amostra tem 47 dados, o Teorema Central do Limite garante que a média amostral segue aproximadamente uma distribuição normal. Por isso, é seguro aplicar o teste de hipóteses usando a média.

\textbf{Resposta (a):}
\[
\boxed{\text{Não é plausível que a população seja normal, mas para } n = 47, \text{ o teste de hipóteses sobre a média é válido}}
\]

\textbf{b. Teste de hipóteses para o consumo médio}

Hipóteses:
\begin{itemize}
  \item $H_0: \mu \leq 200$
  \item $H_1: \mu > 200$
\end{itemize}

Cálculo da estatística do teste:
\[
t = \frac{215 - 200}{235 / \sqrt{47}} = \frac{15}{34{,}27} \approx 0{,}438
\]

Valor crítico:
\begin{itemize}
  \item Graus de liberdade: $df = 46$
  \item Valor crítico (unicaudal, $\alpha = 0{,}10$): $t_{0{,}10; 46} \approx 1{,}30$
\end{itemize}

Como $t = 0{,}438 < 1{,}30$, não rejeitamos $H_0$.

Valor $p$:
\[
p \approx 0{,}33
\]

\textbf{Resposta (b):}
\[
\boxed{\text{Não, os dados não contrariam a opinião anterior ($p \approx 0{,}33$). O consumo médio pode ser } \leq 200 \text{ mg}}
\]





\quest{70}

Dados: $n = 20$, $\bar{x} = 9{,}872$ s, $s = 0{,}10$ s, $\mu_0 = 9{,}75$ s, $\alpha = 0{,}05$

Hip\'oteses:
\begin{itemize}
  \item $H_0: \mu \leq 9{,}75$
  \item $H_1: \mu > 9{,}75$
\end{itemize}

Como $n = 20$ e os dados s\~ao aproximadamente normais, usamos o teste $t$:
\[
t = \frac{\bar{x} - \mu_0}{s / \sqrt{n}} = \frac{9{,}872 - 9{,}75}{0{,}10 / \sqrt{20}} \approx 5{,}45
\]

Valor cr\'itico: $t_{0{,}05; 19} \approx 1{,}729$

Decis\~ao: $t = 5{,}45 > 1{,}729$, ent\~ao rejeitamos $H_0$.

Resposta final:
\[
\boxed{\text{Sim, h\'a evid\^encia de que o tempo m\'edio excede 9{,}75 s ($p < 0{,}0001$)}}
\] 

\quest{71}

\textbf{a)}

Dados: $n = 800$, $x = 16$, $\hat{p} = 0{,}02$, $p_0 = \frac{1}{75} \approx 0{,}0133$, $\alpha = 0{,}05$

Hipóteses:
\begin{itemize}
  \item $H_0: p = 0{,}0133$
  \item $H_1: p \neq 0{,}0133$
\end{itemize}

Verificamos que $np_0 \approx 10{,}64$ e $n(1 - p_0) \approx 789{,}36$, então podemos usar aproximação normal.

Estatística do teste:
\[
Z = \frac{\hat{p} - p_0}{\sqrt{\frac{p_0(1 - p_0)}{n}}} \approx \frac{0{,}02 - 0{,}0133}{0{,}00405} \approx 1{,}64
\]

Valor crítico: $z_{0{,}025} = \pm 1{,}96$

Decisão: Como $|Z| = 1{,}64 < 1{,}96$, não rejeitamos $H_0$.

Resposta final:
\[
\boxed{\text{a. } z = 1{,}64 < 1{,}96, \text{ portanto não rejeitamos } H_0. \text{ Tipo II}}
\]

\textbf{b)}

Valor $p$ aproximado: $2 \times P(Z > 1{,}64) \approx 0{,}10$

Como $p = 0{,}10 < 0{,}20$, rejeitamos $H_0$ para $\alpha = 0{,}20$.

Resposta final:
\[
\boxed{\text{b. Valor } p \approx 0{,}10. \text{ Para } \alpha = 0{,}20, \text{ rejeitamos } H_0. \text{ Sim}}
\]

\quest{73}

Dados: $n = 45$, $\bar{x} = 3107$ psi, $s = 188$ psi, $\mu_0 = 3200$ psi, $\alpha = 0{,}001$

Hipóteses:
\begin{itemize}
  \item $H_0: \mu = 3200$
  \item $H_1: \mu < 3200$
\end{itemize}

Como $n = 45$, usamos o teste $t$:
\[
t = \frac{\bar{x} - \mu_0}{s / \sqrt{n}} = \frac{3107 - 3200}{188 / \sqrt{45}} \approx \frac{-93}{28{,}03} \approx -3{,}32
\]

Valor crítico: $t_{0{,}001; 44} \approx -3{,}26$

Decisão: $t = -3{,}32 < -3{,}26$, então rejeitamos $H_0$.

Resposta final:
\[
\boxed{\text{Sim, h\'a forte evid\^encia de que a m\'edia < 3200 psi ($p \approx 0{,}0007$)}}
\]

\quest{74}

Dados: $n = 72$, $x = 42$, $\hat{p} = \frac{42}{72} = 0{,}5833$, $p_0 = 0{,}75$

Hipóteses:
\begin{itemize}
  \item $H_0: p = 0{,}75$
  \item $H_1: p < 0{,}75$
\end{itemize}

Verificação das condições:
\begin{itemize}
  \item $np_0 = 72 \times 0{,}75 = 54 \geq 10$
  \item $n(1 - p_0) = 72 \times 0{,}25 = 18 \geq 10$
\end{itemize}

Estatística do teste:
\[
Z = \frac{\hat{p} - p_0}{\sqrt{\frac{p_0(1 - p_0)}{n}}} = \frac{0{,}5833 - 0{,}75}{\sqrt{\frac{0{,}75 \times 0{,}25}{72}}} = \frac{-0{,}1667}{0{,}0510} \approx -3{,}27
\]

Valor $p$ (unicaudal):
\[
P(Z \leq -3{,}27) \approx 0{,}00054
\]

Decisão: Como $p < 0{,}05$, rejeitamos $H_0$.

Resposta final:
\[
\boxed{\text{Sim, h\'a forte evid\^encia de que a propor\c{c}\~ao < 0{,}75 ($p \approx 0{,}00054$)}}
\]


\quest{76}

Dados: $n = 32$, $\bar{x} = 17{,}5$ min, $s = 2{,}2$ min, $\mu_0 = 15$ min, $\alpha = 0{,}05$

Hip\'oteses:
\begin{itemize}
  \item $H_0: \mu \leq 15$
  \item $H_1: \mu > 15$
\end{itemize}

Como a distribui\c{c}\~ao \'{e} aproximadamente normal, usamos a estat\'istica:
\[
t = \frac{\bar{x} - \mu_0}{s / \sqrt{n}} = \frac{17{,}5 - 15}{2{,}2 / \sqrt{32}} \approx \frac{2{,}5}{0{,}3889} \approx 6{,}43
\]

Valor $P$: Para $t = 6{,}43$ e $gl = 31$, $P < 0{,}0001$

Decis\~ao: Como $P < 0{,}05$, rejeitamos $H_0$.

Resposta final:
\[
\boxed{\text{Sim, os dados deixam d\'uvida na alega\c{c}\~ao da empresa, pois o valor $P$ \'{e} menor que 0{,}05}}
\]

\end{document}
