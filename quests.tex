% Todas as linhas precedidas pelo simbolo '%' são comentários
% e não afetam em nada o seu texto final.

% IGNORE. Pacotes necessários e acessórios para o documento
\documentclass[12pt]{article}
\usepackage{amsthm}
\usepackage{libertine}
\usepackage[margin=1in]{geometry}
\usepackage{amsmath, amsfonts}
\usepackage{amssymb}
\usepackage{multicol}
\usepackage[brazil]{babel}
\usepackage[shortlabels]{enumitem}
\usepackage{xcolor}
\usepackage{fullpage}
\usepackage{tikz}
\usepackage{enumitem}
\usepackage{fontspec}
\usepackage{unicode-math}
\usetikzlibrary{shapes.geometric, calc}
% ---

\newcommand{\quest}[1]{\section*{Questão #1}} % comando para "criar" uma questão
\setlist[enumerate,1]{label=\textbf{\alph*)}} % faz com que o enumerate enumere em ordem alfabética por padrão.

\begin{document}
\begin{table}[]
\centering
\begin{tabular}{lc}
\hline
\textbf{Nome}                  & \textbf{Matrícula}\\
Carlos Daniel Rodrigues        &  566429           \\
Jones                          &                   \\
Kleberson                      &                   \\
Murilo Vitoriano Alves Fragoso & 570701            \\ \hline
\end{tabular}
\end{table}

\section{Cap. 7 - Exercícios complementares}
\quest{47}
$n = 48, \sum{x_i}= 387.8 \text{ e } \sum{x_i^2=4247.08}$ 
\begin{enumerate}
    \item $\overline{x} = \frac{\sum{x_i}}{n} = \frac{387.8}{48} = 8.08$\\
          $s^2 = \frac{\sum{x_i^2} - \frac{\sum{x_i}^2}{n}}{n-1} = \frac{4247.08 - \frac{387.8^2}{48}}{47} = \frac{4247.08-3133.10}{47} = \frac{1114}{47} \approxeq 23.70$\\
          $s = \sqrt{s^2} = \sqrt{23.70} \approxeq 4.87$\\
          Considerando um nível de confiança de 95\%, temos que $\alpha = 0.05$, podemos fazer:
          \begin{align*}
              IC &= \overline{x} \pm Z_{\frac{0.05}{2}} \cdot \frac{s}{\sqrt{n}}\\
                 &= 8.08 \pm 1.96 \cdot \frac{4.87}{\sqrt{48}}\\
                 &\approxeq \left (6.702; 9.457 \right)
          \end{align*}
    \item Como há 13 valores que excedem 10, a proporção é dada por:\\
    $\hat{p} \approxeq \frac{13}{48} = 0.271$\\
    $\hat{q} = 1 - \hat{p} \approxeq 0.729$\\
    Como queremos um IC de 95\%, temos que $\alpha = 0.05$, logo:
    \begin{align*}
        IC &= \hat{p} \pm Z_{0.05/2} \cdot \sqrt{\frac{\hat{p}\hat{q}}{n}}\\
            &= 0.271 \pm 1.96 \cdot \sqrt{\frac{0.1975}{48}}\\
            &\approxeq (0.145;0.397)
    \end{align*}
\end{enumerate}

\quest{48}
Dados: $n=9$, $\bar{x}=188,0$, $s=7,2$. Intervalo de confiança de 98\%, logo $\alpha=0,02$.

\begin{enumerate}
    \item Como $n < 30$ e não conhecemos o desvio padrão populacional, usamos a distribuição $t$ de Student.
    \item Graus de liberdade: $gl = n-1 = 9-1=8$.
    \item Para $\alpha/2=0,01$ e $gl=8$, consultando a tabela temos:
    \[
    t_{0,01;8} \approx 2,896
    \]
    \item Calculamos o erro padrão:
    \[
    EP = \frac{s}{\sqrt{n}} = \frac{7,2}{\sqrt{9}} = \frac{7,2}{3} = 2,4
    \]
    \item Intervalo de confiança:
    \begin{align*}
        IC &= \bar{x} \pm t_{\alpha/2,gl} \cdot EP \\
           &= 188,0 \pm 2,896 \cdot 2,4 \\
           &= 188,0 \pm 6,95
    \end{align*}
    \item Assim, temos:
    \[
    \text{Limite inferior} = 188,0 - 6,95 = 181,05
    \]
    \[
    \text{Limite superior} = 188,0 + 6,95 = 194,95
    \]
\end{enumerate}

\textbf{Resposta final:}
\[
\boxed{(181,05 \;\;\text{até}\;\; 194,95)}
\]

Com 98\% de confiança, o índice cardíaco médio real dos triatletas durante a natação está entre aproximadamente 181,05 e 194,95 batimentos por minuto.

\quest{49}

Dados: $n=18$ e as observações (ordenadas) são:
\[
22,0, 23,5, 31,5, 32,5, 33,5, 34,0, 35,7, 36,4, 37,2, 39,3, 41,4, 42,5, 44,5, 45,6, 46,7, 46,9, 51,2, 51,5
\]

\begin{enumerate}
    \item \textbf{Boxplot e comentários:}
    
    \begin{itemize}
        \item Primeiro quartil: $Q_1 = 33,5$
        \item Mediana: $Q_2 = 38,25$
        \item Terceiro quartil: $Q_3 = 45,6$
        \item Intervalo interquartil: $IQR = Q_3 - Q_1 = 12,1$
        \item Limites para outliers:
        \[
        LI = Q_1 - 1,5 \times IQR = 33,5 - 18,15 = 15,35
        \]
        \[
        LS = Q_3 + 1,5 \times IQR = 45,6 + 18,15 = 63,75
        \]
        \item Não há valores fora desses limites, portanto sem outliers.
        \item Comentário: Parece haver uma ligeira inclinação positiva na metade do meio da amostra, mas as suíças inferiores são mais longas do que as superiores. A extensão da variabilidade é muito substancial, embora não haja outliers.
    \end{itemize}
    
    \item \textbf{Normalidade:}
    
    A partir da marcação de probabilidade normal, o padrão dos pontos é razoavelmente linear, indicando que é plausível que a amostra venha de uma distribuição normal.
    
    \item \textbf{Intervalo de confiança de 98\% para a média:}
    
    Calculamos a média amostral:
    \[
    \bar{x} = \frac{\sum x_i}{n} = \frac{695,9}{18} \approx 38,66
    \]
    
    Calculamos o desvio padrão amostral:
    \[
    s = \sqrt{\frac{\sum x_i^2 - \frac{\left(\sum x_i\right)^2}{n}}{n-1}} = \sqrt{\frac{28123,59 - \frac{(695,9)^2}{18}}{17}} \approx 8,46
    \]
    
    Graus de liberdade:
    \[
    gl = n - 1 = 17
    \]
    
    Valor crítico da distribuição $t$ para 98\% e $gl=17$:
    \[
    t_{0,01;17} \approx 2,567
    \]
    
    Erro padrão da média:
    \[
    EP = \frac{s}{\sqrt{n}} = \frac{8,46}{\sqrt{18}} \approx 1,993
    \]
    
    Construção do intervalo de confiança:
    \[
    IC = \bar{x} \pm t_{0,01;17} \times EP = 38,66 \pm 2,567 \times 1,993 = 38,66 \pm 5,11
    \]
    
    \[
    IC = (33,55; \; 43,77)
    \]
    
    \textbf{Resposta final:}
    \[
    (33,53; \; 43,79)
    \]
    
\end{enumerate}

\quest{50}

Foi dado um IC de 95\% para a frequência natural média real:
\[
(229,764 ; \; 233,504)
\]
e tamanho da amostra $n=5$.

\begin{enumerate}
    \item \textbf{Encontrando a média amostral:}
    \[
    \bar{x} = \frac{229,764 + 233,504}{2} = 231,634
    \]
    
    \item \textbf{Amplitude e erro máximo no IC de 95\%:}
    \[
    \text{Amplitude} = 233,504 - 229,764 = 3,74
    \]
    \[
    E_{95\%} = \frac{3,74}{2} = 1,87
    \]
    
    \item \textbf{Usando o valor crítico de $t$ para 95\% para encontrar $s$:}
    
    Graus de liberdade: $gl = n - 1 = 4$
    
    \[
    t_{0,025;4} \approx 2,776
    \]
    
    O erro máximo é dado por:
    \[
    E_{95\%} = t_{0,025;4} \cdot \frac{s}{\sqrt{n}}
    \]
    \[
    1,87 = 2,776 \cdot \frac{s}{\sqrt{5}}
    \]
    \[
    \frac{s}{\sqrt{5}} = \frac{1,87}{2,776} \approx 0,6737
    \]
    \[
    s = 0,6737 \cdot \sqrt{5} \approx 0,6737 \cdot 2,236 \approx 1,506
    \]
    
    \item \textbf{Calculando o novo IC de 99\%:}
    
    Nível de confiança 99\% $\rightarrow \alpha=0,01$, logo $\alpha/2=0,005$.
    
    \[
    t_{0,005;4} \approx 4,604
    \]
    
    Erro máximo:
    \[
    E_{99\%} = t_{0,005;4} \cdot \frac{s}{\sqrt{5}} = 4,604 \cdot 0,6737 \approx 3,10
    \]
    
    \item \textbf{Intervalo de confiança de 99\%:}
    \[
    IC_{99\%} = \bar{x} \pm 3,10 = 231,634 \pm 3,10
    \]
    \[
    = (228,534 ; \; 234,734)
    \]
\end{enumerate}

\textbf{Resposta final:}
\[
\boxed{(228,534 \;\text{até}\; 234,734)}
\]


\end{document}
