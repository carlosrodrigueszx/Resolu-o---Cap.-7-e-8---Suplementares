% Todas as linhas precedidas pelo simbolo '%' são comentários
% e não afetam em nada o seu texto final.

% IGNORE. Pacotes necessários e acessórios para o documento
\documentclass[12pt]{article}
\usepackage{amsthm}
\usepackage{libertine}
\usepackage[margin=1in]{geometry}
\usepackage{amsmath, amsfonts}
\usepackage{amssymb}
\usepackage{multicol}
\usepackage[brazil]{babel}
\usepackage[shortlabels]{enumitem}
\usepackage{xcolor}
\usepackage{fullpage}
\usepackage{tikz}
\usepackage{enumitem}
\usepackage{fontspec}
\usepackage{unicode-math}
\usetikzlibrary{shapes.geometric, calc}
% ---

\newcommand{\quest}[1]{\section*{Questão #1}} % comando para "criar" uma questão
\setlist[enumerate,1]{label=\textbf{\alph*)}} % faz com que o enumerate enumere em ordem alfabética por padrão.

\begin{document}
\begin{table}[]
\centering
\begin{tabular}{lc}
\hline
\textbf{Nome}                  & \textbf{Matrícula}\\
Carlos Daniel Rodrigues        &  566429           \\
Jones                          &                   \\
Kleberson                      &                   \\
Murilo Vitoriano Alves Fragoso & 570701            \\ \hline
\end{tabular}
\end{table}

\section{Cap. 7 - Exercícios complementares}
\quest{47}
$n = 48, \sum{x_i}= 387.8 \text{ e } \sum{x_i^2=4247.08}$ 
\begin{enumerate}
    \item $\overline{x} = \frac{\sum{x_i}}{n} = \frac{387.8}{48} = 8.08$\\
          $s^2 = \frac{\sum{x_i^2} - \frac{\sum{x_i}^2}{n}}{n-1} = \frac{4247.08 - \frac{387.8^2}{48}}{47} = \frac{4247.08-3133.10}{47} = \frac{1114}{47} \approxeq 23.70$\\
          $s = \sqrt{s^2} = \sqrt{23.70} \approxeq 4.87$\\
          Considerando um nível de confiança de 95\%, temos que $\alpha = 0.05$, podemos fazer:
          \begin{align*}
              IC &= \overline{x} \pm Z_{\frac{0.05}{2}} \cdot \frac{s}{\sqrt{n}}\\
                 &= 8.08 \pm 1.96 \cdot \frac{4.87}{\sqrt{48}}\\
                 &\approxeq \left (6.702; 9.457 \right)
          \end{align*}
    \item Como há 13 valores que excedem 10, a proporção é dada por:\\
    $\hat{p} \approxeq \frac{13}{48} = 0.271$\\
    $\hat{q} = 1 - \hat{p} \approxeq 0.729$\\
    Como queremos um IC de 95\%, temos que $\alpha = 0.05$, logo:
    \begin{align*}
        IC &= \hat{p} \pm Z_{0.05/2} \cdot \sqrt{\frac{\hat{p}\hat{q}}{n}}\\
            &= 0.271 \pm 1.96 \cdot \sqrt{\frac{0.1975}{48}}\\
            &\approxeq (0.145;0.397)
    \end{align*}
\end{enumerate}

\quest{48}
\end{document}